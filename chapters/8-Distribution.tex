\chapter{Distribution}
CLI - commandline interface only
Focus on theoretical work 
And specifics regarding CUDA considerations regarding its limitations
MAYBE 1 page
Dependencies
Linux vs Windows



\section{Standalone Precompiled Binary}
CLI - commandline interface only



\section{Integration with Legacy Code (GigaMesh)}
CLI - commandline interface or GUI - graphical user interface are possible
\begin{enumerate}
\item Make GigaMesh CUDA aware
	\begin{enumerate}
	\item Update Makefile to include nvcc compiler and new source files
	\item second item
	\end{enumerate}
\item second item
\end{enumerate}
Before CUDA 5.0, if a programmer wanted to call particle::advance() from a CUDA 
kernel launched in main.cpp, the compiler required the main.cpp compilation unit 
to include the implementation of particle::advance() as well any subroutines it 
calls (v3::normalize() and v3::scramble() in this case). In complex C++ 
applications, the call chain may go deeper than the two-levels that our example 
illustrates. Without device object linking, the developer may need to deviate 
from the conventional application structure to accommodate this compiler 
requirement. Such changes are difficult for existing applications in which 
changing the structure is invasive and/or undesirable.~\cite{Cuda14}
In order to use the full functionality of GigaMesh a batch program has to be run 
for generating feature vectors. These vectors contain additional information per 
vertex concerning surface and volume of a set of spheres intersecting the mesh. 
See [MKJB10] for more background information on the Multi Scale Integral 
Invariant (MSII) filtering technique. This operation is rather time consuming 
(it takes hours or even days of computing time) and therefore better runs 
without graphical user interface. Although generating feature vectors is quite 
robust against solo vertices, singularities, non-manifolds and holes, you should 
first clean up your mesh data to get a proper result. So switch to the advanced 
task of polishing your mesh in section 4.1 and return to this section when you 
have got a cleaned mesh. If you do not want to manipulate your mesh you may 
continue directly. Open a terminal and type and change to the mesh-folder by 
typing cd GigaMesh/mesh (note that GigaMesh stands for the GigaMesh installation 
folder). Then start the program nohup ./meshgeneratorfeaturevectors25d\_threads 
-f [-r 2] \& and use nohup at the beginning of the command and \& at the end to 
ensure that the job runs in the background. This is because this step can take 
several hours and you do not want to block the terminal.~\cite[p.~19]{Giga17}



\section{Summary}
Lorem ipsum dolor sit amet, consectetur adipiscing elit. Morbi tincidunt eget 
ipsum eu iaculis. Cras vel sem eu velit eleifend porta vel sit amet massa. Etiam 
a posuere nunc. Aenean aliquam viverra dapibus. Aliquam ac eros a purus feugiat 
rhoncus. Donec faucibus ut nibh ut cursus. Aliquam erat volutpat. Proin efficitur 
nulla sit amet iaculis condimentum. Cras placerat leo vitae venenatis feugiat. In 
hac habitasse platea dictumst. Orci varius natoque penatibus et magnis dis 
parturient montes, nascetur ridiculus mus. In aliquet sagittis dui eu pulvinar. 
Morbi a arcu eu dolor sagittis varius. Aliquam dignissim tortor sed tortor 
suscipit, eget imperdiet mauris convallis.
