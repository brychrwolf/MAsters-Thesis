\chapter{Introduction \& Motivation}
\label{ch1}
Motivated by the increasing importance of \tdd{} acquisition for both industry and academia, and the inherent difficulties associated with processing that data in an efficient way, this thesis presents the parallel variant of \Fors{t} as a meaningful solution.

The work presented in this thesis was conducted as part of the Heidelberg University Excellence Initiative as part of the \gls{FCGL}at the \gls{IWR}. The algorithms presented may be implemented directly as source code, as was already done in CUDA enchanced C++ in order to generate the synthetic \tdd{} and conduct the experiments used for analysis. The serial algorithm is also readily available as part of the GigaMesh framework, which is maintained by the \gls{FCGL}.

%
%
%
%
%
%
\section{Motivation}
%3D-Data is important to many fields
From industrial quality scanning~\cite{ILATO14}, to computerized analysis of documents and artifacts within the Digital Humanities~\cite{Bogacz15}, the demand for high definition \tdd{} is only increasing. And in light of recent tragedies befalling global physical archives, for example the museum fire which destroyed Brazil's oldest museum and its 20 million artifacts~\cite{Andreoni18}, or the numerous war zones occupying archaeologically important sites, 3D-scanning can only become more important.

%3D-Data Noises in data, so must still process
However, raw \tdd{} is typically not suitable for analysis~\cite[p.~25-32]{Mara12}, so processing and pre-processing with a smoothing filter can be necessary. In dense, high-resolution meshes for example, one can see noise propagating as jagged outlines in segment boundaries of connected components when visualizing the output of filters designed for analysis, like \acrlong{tMSIIf} (MSII)~\cite[s.~3.2]{Mara17}.

%processing is difficult
One major complication slowing the development of filters for \tdd{} has been that the window size of any filter must remain static for the duration of its convolution in order for the output response to be mapped correctly back onto the input field~\cite[p.~106-112]{Jaehne97}. While it is trivial define a static-sized filter for convolving regular meshes like raster images, it is a complex and complicated task to create the same for convolving acquired \tdd{}, whose one-ring neighborhoods are subsets of non-planar meshes embedded in $\bR{3}$, uniformly irregular, with completely arbitrary shapes, sizes, and counts of members~\cite[p.~29]{Mara12}~\cite[s.~3.2]{Mara17}.

%3D-Data is big. serial is slow
A problem also arises, that with high-resolution 3D-scanning, comes large amounts of data; commonly totaling millions, or tens of millions of points per scanned item and featuring several hundred points per mm$^2$~\cite[25,144]{Mara17}~\cite[4]{ILATO14}. At that scale, serial algorithms which process that data can no longer be included in the regular workflow of a scientist analyzing the artifacts, because each operation can easily take hours or days to complete.


%gpgpus are cheap, so do the thing
Fortunately, with the introduction of \glspl{GPGPU} to the commercial market, an opportunity to exploit the parallel processing power of \gls{SIMD} systems at a budget obtainable by individual research groups has presented itself. Therefore, the motivation for designing a smoothing filter, which can efficiently convolve over large, irregular, acquired \tdd{} by utilizing commercially available GPGPUs was realized, and thus came \Fors{t}, and the research presented in this thesis.

%
%
%
%
%
%
\section{Related Work}
%Other Mesh filters
As filtering noise from acquired \tdd{} is a motivating topic, adapting filters designed for regular two-dimensional meshes for use with irregular, non-planar, acquired \tdd{} is a topic of current research. One such filter for de-noising meshes while preserving sharp-edges, which was recently adapted for use with point clouds and parallel processing, is the ``The Bilateral Filter for Point Clouds''~\cite{Digne17}. 

%2D filters
Because it is possible to adapt a filter, that was designed for two-dimensional data, to be convolved on \tdd{}, the entire field of digital image processing presents itself as opportunities for related work, for example the filters for feature extraction including smoothing~\cite[299]{Jaehne97}, edge~\cite[331]{Jaehne97} and motion~\cite[397]{Jaehne97} detection, as well as more complicated topics, such as three-dimensional face recognition from two-dimensional images~\cite{faceRecog19}.

%Numerics and Numerical Stability}
As a consequence of the design of the \fors{t}, larger filter responses are only obtained through more convolutions of the filter. Also, as smoothing filters are also characterized as a diffusion-reaction system~\cite[474]{Jaehne97}, the on-going research on the topic of numerical optimization of such systems, especially in regards to the stability of algorithms remains relevant.

%Other approaches to acceleration}
In our research, the parallel variant of \fors{t} was implemented using the proprietary CUDA framework~\cite{CUDA18}, however, there exists another framework which enables the use of non-NVIDIA produced \glspl{GPGPU} for parallel processing. It is called OpenCL, and it is developed by The Khronos Group~\cite{Khronos19} to enable the general processing on most \glspl{GPU}, regardless of its manufacturer.

Other frameworks for developing software for parallel processing which are not used in this thesis include: OpenMP~\cite{OpenMP19} which enable the use of distributed computation across multiple machines and networks, and POSIX Threads, or pthreads~\cite[195-210]{Lang17}, for using the \gls{SIMD} environment made available by all modern day \glspl{CPU}.

%
%
%
%
%
%
\section{Structure this Thesis}
This document is structured into the following chapters: Chapter 2 briefly covers many theories, concepts, and frameworks across many fields of study in order to focus on a few specific topics which have a direct influence on research presented this thesis. These include topics from set theory, linear algebra, geometry, and topology. \Fors{T} is presented in Chapters 3, 4, and 5, with Chapter 3 introducing the mathematical foundations upon which the filter was designed, Chapter 4 defining the serial algorithm for implementing the filter, and Chapter 5 analyzing the serial filter in order to present the parallel variant of \fors{t} algorithm. The example meshes used in experimentation are described in Chapter 6, before both the filter response as well as the performance of the parallel algorithm are evaluated. Finally a conclusion for this thesis and an outlook for future enhancements is given in Chapter 6.
