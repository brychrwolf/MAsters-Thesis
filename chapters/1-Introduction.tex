\chapter{Introduction \& Motivation}
\label{ch1}
Motivated by the increasing importance of \tdd{} acquisition for both industry and academia, and the inherent difficulties associated with processing that data in an efficient way, this thesis presents the parallel variant of \Fors{t} as a meaningful solution.

The work presented in this thesis was conducted with support from the Heidelberg University Excellence Initiative through the DFG, as well as the \gls{FCGL} in cooperation with the Graduate School for Mathematical and Computational Modeling (HGS MathComp) at the \gls{IWR}. The algorithms presented may be implemented directly as source code, as was already done in CUDA enhanced C++, in order to generate the synthetic \tdd{} and conduct the experiments used for analysis. The serial algorithm is also readily available as part of the GigaMesh framework, which is maintained by the \gls{FCGL}.

%
%
%
%
%
%
\section{Motivation}
%3D-Data is important to many fields
From industrial quality scanning~\cite{ILATO14}, to computerized analysis of documents and artifacts within the Digital Humanities~\cite{Bogacz15}, the demand for high-definition \tdd{} is only increasing. And in light of recent tragedies befalling global physical archives, like the museum fire which destroyed Brazil's oldest museum and its 20 million artifacts~\cite{Andreoni18}, or the numerous war zones occupying archaeologically important sites, 3D-scanning will continue to increase in importance.

%3D-Data Noises in data, so must still process
However, raw \tdd{} is typically not suitable for analysis~\cite[p.~25-32]{Mara12}, so processing and pre-processing with a smoothing filter becomes necessary. In dense, high-resolution meshes, for example, one can see noise propagating as jagged outlines in segment boundaries of connected components when visualizing the output of filters designed for analysis, like \acrlong{tMSIIf} (MSII)~\cite[s.~3.2]{Mara17}.

%processing is difficult
One major complication slowing the development of filters for \tdd{} has been that the window size of any filter must remain static for the duration of its convolution, in order for the output response to be mapped correctly back onto the input field~\cite[p.~106-112]{Jaehne97}. While it is trivial to define a static-sized filter for convolving regular meshes like raster images, it is a complex and complicated task to create the same for convolving acquired \tdd{}, whose one-ring neighborhoods are subsets of non-planar meshes embedded in $\bR{3}$, uniformly irregular, with completely arbitrary shapes, sizes, and counts of members~\cite[p.~29]{Mara12}~\cite[s.~3.2]{Mara17}.

%3D-Data is big. serial is slow
Another problem which arises, is that with high-resolution 3D-scanning, the data output is often comprised of millions, or tens of millions, of points per scanned item and features several hundred points per mm$^2$~\cite[25,144]{Mara17}~\cite[4]{ILATO14}. At that scale, serial algorithms processing that data can no longer be included in the regular workflow of a scientist analyzing the artifacts, because each operation can easily take hours or days to complete.


%gpgpus are cheap, so do the thing
Fortunately, with the introduction of \glspl{GPGPU} to the commercial market, individual research groups now have the opportunity to exploit the parallel processing power of \gls{SIMD} systems, without needing access to an institutional supercomputer. Therefore, the motivation for designing a smoothing filter, which can efficiently convolve over large, irregular, acquired \tdd{} by utilizing commercially available \glspl{GPGPU}, was realized. Thus came \Fors{t}, and the research presented in this thesis.

%
%
%
%
%
%
\section{Related Work}
%Other Mesh filters
As filtering noise from acquired \tdd{} is a motivating topic, the adaptation of filters already established for regular, two-dimensional meshes to be used with irregular, non-planar, acquired \tdd{} is a topic of current research. One such filter for de-noising meshes while preserving sharp-edges is the ``The Bilateral Filter for Point Clouds''~\cite{Digne17}, which presents an adaptation of the bilateral filter two-dimensional images for use with point clouds and parallel processing.

%2D filters
Because it is possible to adapt a filter, that was designed for two-dimensional data, to be convolved on \tdd{}, the entire field of digital image processing presents itself an opportunity for related work. For example the filters for feature extraction include smoothing~\cite[p.~299]{Jaehne97}, edge~\cite[p.~331]{Jaehne97} and motion detection~\cite[p.~397]{Jaehne97}, as well as more complicated topics, such as three-dimensional face recognition from two-dimensional images~\cite{faceRecog19}.

%Numerics and Numerical Stability}
As a consequence of the design of the \Fors{t}, larger filter responses are only obtained through more convolutions of the filter. Additionally, as smoothing filters are also characterized as a diffusion-reaction system~\cite[p.~474]{Jaehne97}, the on-going research on the topic of numerical optimization of such systems, especially in regards to the stability of algorithms remains relevant.

%Other approaches to acceleration}
In our research, the parallel variant of \fors{t} was implemented using the proprietary CUDA framework~\cite{CUDA17}, however, there exists another framework which enables the use of non-NVIDIA produced \glspl{GPGPU} for parallel processing. It is called OpenCL and it is developed by The Khronos Group~\cite{Khronos19} to enable general processing on most \glspl{GPU}, regardless of manufacturer.

Other frameworks for developing software for parallel processing, which are not used in this thesis include: OpenMP~\cite{OpenMP19} which enables the use of distributed computation across multiple machines and networks, and POSIX Threads, or pthreads~\cite[p.~195-210]{Lang17}, for use in the limited \gls{SIMD} environment, made available by all modern day \glspl{CPU}.

%
%
%
%
%
%
\section{Structure this Thesis}
This document is structured into the following chapters: Chapter 2 briefly covers many theories, concepts, and frameworks across multiple fields of study, in order to focus on a few specific topics which have a direct influence on research presented in this thesis. These include topics from set theory, linear algebra, geometry, and topology. \Fors{T} is presented in Chapters 3, 4, and 5. First, Chapter 3 introduces the mathematical foundations upon which the filter was designed. Next, Chapter 4 defines the serial algorithm for implementing the filter. Then, Chapter 5 analyzes the serial filter in order to present the parallel variant of \fors{t} algorithm. The example meshes used in experimentation are described in Chapter 6, before both the filter response as well as the performance of the parallel algorithm are then evaluated. Finally, the conclusion for this thesis and an outlook for future enhancements is provided in Chapter 6.
