\chapter{Fast One-Ring Smoothing: Parallel Algorithms}
\label{ch6}
In the previous two chapters we presented the mathematical grounding and serial algorithms for an improved version of \Fors{t}, as it is currently implemented within the GigaMesh framework, and while it has improved accuracy in regards to the method of weighting the mean function values, it is still yet entirely serial in design, so that unfortunately, its performance suffers greatly under the complexity of modern mesh sizes, which with the current high resolution scanners in use, can grow to be MESH\_SIZES\todoResearch{mesh sizes}.

In this chapter, we will now explore the serial version of this as-yet-unpublished algorithm, in order to negotiate any instances of control or data dependency, hopefully discovering manifestations of independent procedures worthy of exploiting with parallel processing, with the goal of improving the performance of the filter when implemented on a system capable of parallel computation.
\todoBackground{Data dependencies}

Following the pattern set by the serial algorithm in Chapter~\ref{ch5}, the parallel algorithm for \Fors{t} also has three main parts. In contrast however, the parallel variant of each of the three serial parts are split further into subroutines, in order to delegate blocks of instructions to separate threads of execution, while protecting critical sections, and avoiding data and control dependencies, so that the threads can be safely executed in parallel.

The following three sections will each focus on producing the parallel variant of one of the three parts of the serial algorithm which was presented in Chapter~\ref{ch5}, by first analyzing the serial algorithm through the lens of parallel programing, next defining the strategy behind the design of the parallel variant, then finally providing the pseudo code definitions to implement that part of the algorithm.

%
%
%
%
\section{Build Neighborhoods in Parallel}
\label{ch6sBNP}
The sole purpose of Algorithm~\ref{alg:serialBuildNeighborhoods}, as discussed in detail in Section~\ref{ch5sBN}, is to explore every triangular face in the set $\bT$ in order to generate a family of sets of neighborhoods $\bN$, so that subsequent algorithms can recall and iterate over those associations and negate the computational costs incurred by searching the entire mesh each time the membership of a neighborhood must be known. In addition to that goal, because of the nondeterministic nature of the sum of the cardinalities of each neighborhood of $\bN$, the parallel variant of the $\mathit{buildNeighborhoods}$ procedure must also provide a census, a total count of all neighbors in all neighborhoods, so that the next two parts of the parallel variant of \fors{t} algorithm can accurately predict work loads, and evenly delegate portions of the work to independent threads of execution.

%
%
\subsection{Analysis of Serial Algorithm~\ref{alg:serialBuildNeighborhoods}: Build Neighborhoods}
\label{ch6sBNPssASABN}
In this section, we analyze through the lens of parallel programing, the~\nameref{saBN.title} as presented in Section~\ref{ch5sBN}. The strategy we employ involves reading the serial algorithm line by line, building a dependency graph, and as opportunities for exploiting parallelism are discovered, a strategy\todoReword{strategy twice is bad} for maximizing efficiency is devised.
\todoBackground{Dependency Graph}

Starting in line~\ref{sbn1} of Algorithm~\ref{alg:serialBuildNeighborhoods}, we read the declaration of the function \textit{serialBuildNeighborhoods} requiring the entire set of triangular faces $\bT$ as an input, then already in line~\ref{sbn2}, we encounter a loop over each face $\bt$ in $\bT$. In general, concurrency found in the structure of a loop may be completely exploited by a parallel implementation of the same procedure, but only in the absence of loop-carried dependence. So, in order to determine if the iterations of this loop may be computed in parallel, an analysis of every operation in the loop block is required.\todoResearch{Amdahl's Law}
\todoBackground{loop level parallelism, loop-carried and loop-independent}
\todoBackground{static vs volatile memory}

\todoReword{add subsubsections for initial loop, dependency graph, loop block, parallel algorithm}
Figure~\ref{fig:sabnDataDependencies} illustrates the data dependencies inherent to each iteration of the loop over the faces $\bt$ in $\bT$, as found in lines~\ref{sbn2}~-~\ref{sbn5} of Algorithm~\ref{alg:serialBuildNeighborhoods}.


\begin{figure}[ht]
	\includestandalone[width=0.8\textwidth]{figures/tikz/sabnDataDependencies}
	{\caption[Data Dependencies in Serial Algorithm~\ref{alg:serialBuildNeighborhoods}: Build Neighborhoods]{An illustration of the data dependencies found in Algorithm~\ref{alg:serialBuildNeighborhoods}. Square nodes indicate a values stored in memory, oval nodes indicate operations. Single arrows indicate reading values, double arrows indicate writing values to memory. Drawn in teal color are static memory, and the values and operations which are independent and free of data dependencies. Drawn in coral color, are volatile memory, the values stored there, which may be modified at anytime during a procedure, and the operations which rely on those volatile values.}\label{fig:sabnDataDependencies}}
\end{figure}
\todoReword{mention "due to the commutative property of the union operation, as shown in Equation~\ref{eq:ascAndComPropertiesOfUnions}"}

As denoted by the first three teal colored lines, the first operations are to read the three points $\bp_a$, $\bp_b$, $\bp_c$ from the triangular face $\bt$, stored is the set $\bT$ in static memory, which will never be modified by this, or any other procedure. Also, despite the fact that faces can be adjacent to one another, each face is distinctly defined in the set $\bT$, therefore, each face $\bt$ can be considered independent of each other face. Thus, this first instruction of the loop block is totally free from any control or data dependencies. The next line in the algorithm is a complex instruction, though, so it must be analyzed in parts.

The first part of the instruction on line~\ref{sbn3} uses the point $\bp_a$ as a reference to read the state of $\bN_a$ from the set $\bN$ in volatile memory, as denoted as teal and coral colored lines, respectively. $\bN_a$ is stored in volatile memory, which may potentially be modified by other threads, because it is not known a priori which thread will discover the points which belong to that neighborhood, thus it must be accessible to any and all threads. Next, also drawn in coral color, are the two union operations performed in succession between $\bN_a$ and the point $\bp_b$, then $\bp_c$. While these two operations are themselves independent, being performed only on values currently immutable by other procedures, they do rely on having already read the status of $\bN_a$, which indeed does have its own data dependence, hence the indicative coloring. Finally, illustrated as a coral colored double arrow, the updated set $\bN_a$ is stored back into $\bN$, replacing the original set in volatile memory.

Both reading from, and writing to $\bN_a$ in volatile memory constitutes a critical section in the algorithm, which must be accounted for in the design of the parallel variant of this part of the algorithm. Furthermore, both dependencies lie in a single path of dependence, which means that the entire group of operations must be protected by a locking mechanism as described in Section~\ref{ch2sPPssMS}. 

Fortunately, because both dependencies revolve around the same set $\bN_a$, instead of implementing a global mutex to protect the critical section in all threads computing \wmfv{s}, it will be sufficient to design the algorithm with a set of mutexes, where each mutex only protects a neighborhood's portion of the total critical sections being processed in parallel.

The significance of this design choice comes from how typical of acquired \tdd{}, the ratio of points in $\bP$ to the average neighborhood size approaches $|\bP|$, appreciable with even moderately sized meshes, so instead of all processors, which can count in the thousands, blocking in wait of entering the same critical section for all neighbors of all points, which can count in the tens of millions, each processor must now only monitor the mutex of the neighborhood for which it is working, which typically will include only about 6 neighbors.\todoReword{reference where average neighborhoods size is determined}

Lines~\ref{sbn4}~and~\ref{sbn5} behave similarly to line~\ref{sbn3}, except they concern other neighborhoods, which in turn, need their own mutexes.\todoBackground{path of dependence} Fortunately, as is modeled with a simple mesh in Figure~\ref{fig:unionsOfSimpleBuildNeighborhoods}, the union operation is performed exactly twice per point per face; that is, once each between the neighborhood of the center point and a neighboring point, for a total of six times per face. By exploiting this pattern and executing the two union operations sequentially within a single mutex, one can mitigate exactly half of the possible collisions in the $\mathit{buildNeighborhoods}$ procedure.

%
%
\subsection{Parallel Variant of Build Neighborhoods}
\label{ch6sBNPssPVBN}
Algorithm~\ref{alg:parallelBuildNeighborhoods} defines the structure and procedures required to implement a parallel variant of the function $\mathit{serialBuildNeighborhoods}$, while ensuring correctness by using a set of mutexes to guard critical sections. The main function $\mathit{parallelBuildNeighborhoods}$ requires only the number of available processors $\rho$, the set of all triangular faces $\bT$, and the cardinality of the set of points $|\bP|$ in order to complete its task. Initially, the ``stride'' is calculated in line~\ref{algPBN.stride} as the portion of the total work load, to which a single processor will be instructed to compute, equal to the problem size divided by the number of available processors; here, the unit of work being an entire triangular face $\bt$. Next, a set of mutexes are created, containing an equal amount of elements as the cardinality of points in $\bP$, to be used individually for each neighborhood. \todoReword{remove if no longer setting $\fM$} Then begins the loop to spawn a ``build'' thread for a quarter of the available processors in the system, with the scalar four being a result of each build thread's need to spawn three additional ``union'' threads. Then all the working threads must synchronize first, before the family of sets of neighborhoods $\bN$ can be finally realized.

In line~\ref{algPBN.build}, we read that along with access to the set of triangular faces $\bT$ and set of mutexes $\fM$, each build thread in Algorithm~\ref{alg:parallelBuildNeighborhoods} also requires an index $\Pi$, which it uses along with the stride $sigma$, to calculate the lower and upper boundaries, $\check{\sigma}$ and $\hat{\sigma}$, for the portion of the work load on which it should compute. Next, it iterates over each face $\bt$ within those stride boundaries, and spawns three union threads; one each for the three points comprising the corners of the current triangular face.

Line~\ref{algPBN.safeUnion} shows that each union thread requires not only the family of sets of neighborhoods $\bN$ and the set of mutexes $\fM$, but also the point which it will consider to be central, and the two points which are the central point's neighbors. With those inputs, each union thread then attempts to lock the mutex which shares the index of both the central point and its corresponding neighborhood, blocking if necessary, in order to safely perform the union operation between the currently known set of neighbors for the central point, and the set of its two newly discovered neighbors; finally, saving the updated neighborhood to volatile shared memory while still protected by the mutex, before unlocking the mutex.
\todoBackground{blocking mutex}

\begin{algorithm}[ht]
	\DontPrintSemicolon
	\SetCommentSty{small}
	\SetKwFor{For}{for}{:}{}
	\SetKwProg{Func}{Function}{}{}
	\SetKwProg{Sub}{Subroutine}{}{}
	\SetKwInOut{Input}{Input}\SetKwInOut{Output}{Output}

	\Input{the number of available processors $\rho$, \\
		the set of all triangular faces $\bT$, \\
		the cardinality of the set of points $|\bP|$}
	\Output{the family of sets of discovered neighborhoods $\bN$}

	\bigskip
\nl	\Func{parallelBuildNeighborhoods($\rho$, $\bT$, $|\bP|$)}{
\nl		$\sigma \leftarrow |\bT|\mathbin{/}\rho$\tcc*{assuming an integer quotient}
\nl		$\fM \leftarrow \{\mu_1,\,\ldots,\,\mu_{|\bP|}\}$\;
\nl		\For{$\Pi \leftarrow 1$ \KwTo $\rho\mathbin{/}4$}{
\nl			\ProgSty{$\sim$build($\Pi$, $\sigma$, $\fM$, $\bT$)}\;
		}
\nl		\ProgSty{synchronizeThreads()}\;
	}

	\bigskip
\nl	\Sub{build($\Pi$, $\sigma$, $\fM$, $\bT$)}{\label{algPBN.build}
\nl		$\check{\sigma} \leftarrow (\Pi-1)\,\sigma+1$\tcc*{works through its stride}\label{algPBN.stride}
\nl		$\hat{\sigma} \leftarrow \Pi\,\sigma$\;
\nl		\For(\tcc*[f]{$\bt = \left \{\bp_a, \bp_b, \bp_c\right \}$}){$\bt \in \{\bt_{\check{\sigma}},\ldots,\,\bt_{\hat{\sigma}}\}$}{
\nl			\ProgSty{$\sim$safeUnion($\fM$, $\bN$, $\bp_a$, $\bp_b$, $\bp_c$)}\;
\nl			\ProgSty{$\sim$safeUnion($\fM$, $\bN$, $\bp_b$, $\bp_a$, $\bp_c$)}\;
\nl			\ProgSty{$\sim$safeUnion($\fM$, $\bN$, $\bp_c$, $\bp_a$, $\bp_b$)}\;
		}
	}

	\bigskip
\nl	\Sub{safeUnion($\fM$, $\bN$, $a$, $b$, $c$)}{\label{algPBN.safeUnion}
\nl		$\ProcSty{lock}(\mu_a)$\;
\nl		$\bN_a \leftarrow \bN_a \cup \{b,\,c\}$\;
\nl		$\ProcSty{unlock}(\mu_a)$\;
	}
	\caption{Parallel algorithm for building the family of sets of all members of each neighborhood discovered in the mesh \label{alg:parallelBuildNeighborhoods}}
\end{algorithm}%
\nomenclature[]{$\rho$}{the number of available processors}%
\nomenclature[]{$\sigma$}{the ``stride'', the size of a block of work intended for a single processor, equal to the problem size divided by the number of available processors}%
\nomenclature[]{$\check{\sigma}$}{the index of the beginning of a stride}%
\nomenclature[]{$\hat{\sigma}$}{the index of the beginning of a stride}%
\nomenclature[]{$\Pi$}{the index representing a single processor}%
\nomenclature[]{$\fM$}{a set of mutexes}%
\nomenclature[]{$\mu_v$}{a specific mutex}%
\nomenclature[]{$\sim$\textit{process}}{a process to be run in parallel by a new thread}%

%
%
\subsection{Parallel Recursive Census Neighborhoods}
\label{ch6sBNPssPRCN}
The motivation behind Algorithm~\ref{alg:parallelRecursiveCensusNeighborhoods} is that the next two parts of parallel algorithm for \fors{t} perform individual computations for each neighbor in the family of sets of neighborhoods $\bN$, and so require a total count of all neighbors in all neighborhoods in order to accurately predict and evenly delegate portions of the work load to each independent thread of execution. Additionally, because of the nondeterministic nature of the cardinality of neighborhoods in acquired \tdd{}, therefore also the total sum of all cardinalities of neighborhoods in $\bN$, the parallel variant of the function $\mathit{serialBuildNeighborhoods}$ must also provide a census of total membership, defined as
%
\begin{equation}
	\hat{n} := \sum\limits_{v=1}^{|\bN|}{|\bN_v|}
	\label{eq:censusNeighborhoods}
\end{equation}%
\nomenclature[]{$\hat{n}$}{census total of all neighbors in all neighborhoods in $\bN$}%

While a sum of all cardinalities could be accumulated by each thread in  Algorithm~\ref{alg:parallelBuildNeighborhoods}, unlike in the serial version, it would very inefficient\footnote{It is possible that in some programming languages, the underlying data structure may automatically keep a record of the size of membership in $\bN$. In that case, one would simply ignore Algorithm~\ref{alg:parallelRecursiveCensusNeighborhoods} and instead assign that generated value to $\hat{n}$.} due to the added overhead of collisions with the locking mechanism required to protect the shared value of the total sum.  Alternatively, algorithm~\ref{alg:parallelRecursiveCensusNeighborhoods} only requires as inputs, the family of sets of neighborhoods $\bN$, and the number of processors available in the system, but is highly parallel and performs at the quick rate of $O(log_2(n))$, so when using a system with multiple processors, it is a much more efficient alternative to calculating the sum implicitly while building $\bN$.

\begin{figure}[ht]
	{\includegraphics[width=1.0\linewidth]{example-image-16x9.png}}
	%\includestandalone[width=0.8\textwidth]{figures/tikz/sabnDataDependencies}
	{\caption[wefwefwef]{sdfsdf}\label{fig:wfwef}}
\end{figure}

As Algorithm~\ref{alg:parallelRecursiveCensusNeighborhoods} is a recursive algorithm, it is best described by dividing it into its three distinct parts. The first part, ``the termination clause'', only executes when the cardinality of the current subset of $\bN$ is two or less\footnote{This conditional operator allows for unit subsets of only a single member, which is possible anytime one processes a set with a cardinality other than one of the positive powers of two, \{2, 4, 8,\ldots , 1048579, \ldots\}, so most of the time.}, then finally returning the sum\footnote{This ignores the trivial case where the cardinality of the full family of sets $\bN$ is less than three, thus never being processed by the \textit{parallelSum} subroutine. If that is something required, only a simple modification similar to the conditional statement found in line~\ref{algPRCN.onlyFirstCall} must be made.} of the members as $\hat{n}$. In all other cases, when the cardinality of the current subset of $\bN$ is greater than two, the second part, ``working towards the termination state'', begins.
\todoBackground{recursive algorithms, three parts}

The second part of this recursive strategy describes summing in parallel, the cardinality or value of each adjacent pair of members in the current subset of $\bN$, then saving the sum of each addition in a new ordered subset of integers $\widetilde{\bN}$, which will have half the cardinality of the current subset. To that end, the stride is calculated using one half of the cardinality of $\bN$ as the total work load, to be divided among all of the available processors. Next, a new thread is spawned for each processor in order to execute the subroutine \textit{parallelSum} to process each portion of the work load in parallel.

The subroutine \textit{parallelSum} requires as inputs: a process index $\Pi$, the stride $sigma$, and the current subset of neighborhood cardinalities being processed\footnote{or if this is the first call, the family of sets of neighborhoods $|\bN|$}. First it calculates the lower and upper boundaries, $\check{\sigma}$ and $\hat{\sigma}$, for the portion of the work load on which it should compute. Next, it iterates over every-other index within the stride boundries, in order to return the sum of the cardinalities of each even-odd pair of neighborhoods. While it may appear that this is an accumulation threatening a race-condition, because each sum is stored as its own unique value to be used in subsequent additions, albeit in the same set, there is no need to protect these operations with locking mechanisms, thus contributing greatly to the overall efficiency of the algorithm.

In the third and final part, ``the recursive call'', the function \textit{parallelRecursiveCensusNeighborhoods} must first wait for all the threads executing instances of \textit{parallelSum} to synchronize before  calling itself, using the new subset $\widetilde{\bN}$ as a parameter instead of the original $\bN$. In each iteration every available processor is utilized, and the cardinality of $\bN$ is reduced by half, quickly approaching the termination clause, while requiring no locking mechanisms, and even without memory recycling, requiring less than twice the total memory required than just storing $\bN|$.

\begin{algorithm}[ht]
	\algotitle{Parallel Algorithm for Recursively Counting a Census of all Neighborhoods}{paRCN.title}
	\DontPrintSemicolon
	\SetCommentSty{small}
	\SetKwFor{For}{for}{:}{}
	\SetKwProg{Func}{Function}{}{}
	\SetKwProg{Sub}{Subroutine}{}{}
	\SetKwInOut{Input}{Input}\SetKwInOut{Output}{Output}

	\Input{the number of available processors $\rho$, \\
		the family of sets of neighborhoods $\bN$}
	\Output{the count of all neighbors in all neighborhoods $\hat{n}$}

	\bigskip
\nl	\Func{parallelRecursiveCensusNeighborhoods($\rho$, $\bN$)}{
\nl		\eIf{$|\bN| \leq 2$}{
\nl			$\hat{n} \leftarrow \sum_{i=1}^{|\bN|}\bN_i$\;
		}{%Else
\nl			$\sigma \leftarrow |\bN|\mathbin{/}(2\,\rho)$\;
\nl			\For{$\Pi \in \{1,\ldots,\rho\}$}{
\nl				\ProgSty{$\sim$parallelSum($\Pi$, $\sigma$, $\bN$)}\;
			}
\nl			\ProgSty{synchronizeThreads()}\;
\nl			\ProgSty{parallelRecursiveCensusNeighborhoods($\rho$, $\widetilde{\bN}$)}\;
		}
	}

	\bigskip
\nl	\Sub{parallelSum($\Pi$, $\sigma$, $\bN$)}{
\nl		$\check{\sigma} \leftarrow (\Pi-1)\,\sigma+1$\;
\nl		$\hat{\sigma} \leftarrow 2\,\Pi\,\sigma$\;
\nl		\For{$v \in \{\check{\sigma},\,\check{\sigma}\sps{}2,\,\check{\sigma}\sps{}4,\ldots,\,\hat{\sigma}\}$}{
\nl			\eIf(\tcc*[f]{only occurs in first call\footnotemark}){$\bN$ is a family of sets}{\label{algPRCN.onlyFirstCall}
\nl				$\widetilde{\bN_v} \leftarrow |\bN_v| + |\bN_{\sxpx{v}{1}}|$\;
			}{%else
\nl				$\widetilde{\bN_v} \leftarrow \bN_v + \bN_{\sxpx{v}{1}}$\;
			}
		}
	}
	\caption{Parallel algorithm for recursively counting a census of all neighbors in all neighborhoods \label{alg:parallelRecursiveCensusNeighborhoods}}
\end{algorithm}%
\nomenclature[]{$\widetilde{\bN}$}{a subset of $\bN$}%
\todoReword{mention that non-existent neighborhoods should be treated as having a cardinality of 0}
\footnotetext{It is important to notice in line~\ref{algPRCN.onlyFirstCall}, the distinction between the family of sets of neighborhoods $\bN$ in the first call of \textit{parallelRecursiveCensusNeighborhoods}, with that of the set of integers accumulating in the subset $\widetilde{\bN}$ in subsequent calls.}

%
%
%
%
\section{Calculating Edge Lengths in Parallel}
\label{ch6sCELP}
The goal of this section is to produce the parallel variant of  Algorithm~\ref{alg:serialCalculateEdgeLengths}, as presented in Chapter~\ref{ch5}, which has the goal of building a set of pre-calculated edge lengths $\bE$, as well as determining the global minimum edge length $\gelm$; both essential parameters of Algorithm~\ref{alg:serialConvolveFilter}. First we will analyze the serial algorithm through the lens of parallel programing, to be followed by defining the strategy behind the design of the parallel variant, then finally providing the pseudo code definitions to implement Algorithm~\ref{alg:parallelCalculateEdgeLengths}.


%
%
\subsection{Analysis of Serial Algorithm~\ref{alg:serialCalculateEdgeLengths}: Calculate Edge Lengths}
\label{ch6sCELPssASACEL}
In this section, we analyze the~\nameref{saCEL.title} as presented in Section~\ref{ch5sCEL}, through the lens of parallel programing. The strategy we employ involves reading the serial algorithm line by line, building a dependency graph, and as opportunities for exploiting parallelism are discovered, a strategy\todoReword{strategy twice is bad} for maximizing efficiency is devised.

Starting in line~\ref{scel1} of Algorithm~\ref{alg:serialCalculateEdgeLengths}, we read the declaration of the function \textit{serialCalculateEdgeLengths} requiring the set of points $\bP$, as well as the family of sets of neighborhoods $\bN$ as inputs. Then, in line~\ref{scel2}, we encounter a loop over each point $\bp_v$ in $\bP$. In order to determine if the iterations of the loop may instead be computed in parallel, an analysis of every operation in the loop block for loop-carried dependencies is required, which in this case, includes lines~\ref{scel3}-~\ref{scel5}.

The first operation in the loop block, line~\ref{scel3}, starts another loop over each point $\bp_i$ in the neighborhood $\bN_v$. Because there are no operations in the first loop which starts on line~\ref{scel2}, that are not also in the nested loop starting at line~\ref{scel3}, we may consider them together as a single multi-dimensional loop; one which iterates over the entirety of the unpredictable dimensions of the nondeterministic family of sets of neighborhoods $\bN$. We will see this pattern again in Algorithm~\ref{alg:parallelConvolveFilter}, and indeed, one can be unroll both loops together in order to compute the loop block in parallel by utilizing the total count of neighbors and the average size of all neighborhoods, barring any occurrences of loop-carried dependencies, of course.

As can be clearly seen in Figure~\ref{fig:neighborhoods}, the cardinality of each individual neighborhood, indeed can not be predicted for irregular, triangle meshes, like those typical of acquired \tdd{}. However, that becomes less important because it is possible to know the total census count of neighbors in all neighborhoods of $\bN$, by actually counting the size of each neighborhood a posteriori, which is is done in Algorithm~\ref{alg:parallelRecursiveCensusNeighborhoods}. With $\hat{n}$, we will in the next chapter, unroll this pair of loops and instruct a calculable number of threads to efficiently process the loop block in parallel.

Figure~\ref{fig:sacelDataDependencies} illustrates the data dependencies inherent to each iteration of the pair of loops over the points $\bp_v$ in $\bP$, and then $\bp_i$ in $\bN_v$, as found in lines~\ref{scel2}~-~\ref{sbn5} of Algorithm~\ref{alg:serialCalculateEdgeLengths}.

\begin{figure}[ht]
	\includestandalone[width=1.0\textwidth]{figures/tikz/sacelDataDependencies}
	{\caption[Data Dependencies in Serial Algorithm~\ref{alg:serialCalculateEdgeLengths}: Calculate Edge Lengths]{An illustration of the data dependencies found in Algorithm~\ref{alg:serialCalculateEdgeLengths}. Square nodes indicate a values stored in memory, oval nodes indicate operations. Single arrows indicate reading values, double arrows indicate writing values to memory. Drawn in teal color are static memory, and the values and operations which are independent and free of data dependencies. Drawn in sand color are mutable memory, the values stored there, and the operations which rely on them. These are independent, however may be modified during processing. Drawn in coral color, are volatile memory, the values stored there, which may be modified at anytime during a procedure, and the operations which rely on those volatile values.}\label{fig:sacelDataDependencies}}
\end{figure}

As denoted by a teal colored lines, the first operation reads the current point $\bp_v$ from $\bP$ in static memory. Next, that point is referenced to read the corresponding neighborhood $\bN_v$ from $\bN$. Then the point $\bp_i$ is determined, and combined with $\bp_v$ in order to compute $\ellstar$, the distance between the two points.

Drawn in sand color, is the procedure to store the resulting independent scalar as $\bE_{\sv{i}}$, in its own unique place in mutable memory\footnote{which requires special attention during implementation, because operations involving reading and writing to mutable memory do not require the expensive protection of locking mechanisms, even though they may be modified concurrently.}. Next, as drawn in coral color, the current global minimum edge length $\gelm$ is read from volatile memory and is compared to $\bE_{\sv{i}}$ from mutable memory, drawn in sand color, finally storing the minimum of the two values as an updated $\gelm$ back into volatile memory.

Line~\ref{scel3} of Algorithm~\ref{alg:serialCalculateEdgeLengths} is the $\ellstar$ operation, the most costly operation performed by \fors{t}, due to the use of the square root operation $\sqrt{(\cdot)}$. Therefore, unlike with the union operation in Algorithm~\ref{alg:parallelBuildNeighborhoods}, it is of paramount importance that we avoid any unnecessary duplication of $\ellstar$.
%In Algorithm~\ref{alg:serialBuildNeighborhoods}, \nameref{alg:serialBuildNeighborhoods},
In the family of sets of neighborhoods, each pair of adjacent points are represented exactly twice, as illustrated by the simple example in Figure~\ref{fig:triangularFaces}, being indexed once from both directions, and while calculating the length both times is exactly what we want to avoid, it is but a design choice related to Section ??\todoReference{memory vs speed}\todoBackground{memory vs speed} whether to actually store the length twice.

The benefit of storing the set of adjacent points twice, once in both directions, it that it creates an implicit reverse lookup-table which is well documented\todoCitation{multiples, reverse lookup table} for increasing the speed of computations, and further simplifies the complexity of indexing the values. If the average number of neighbors per point is six\todoResearch{average number of neighbors}, then the number of edge lengths to be calculated and stored will be six times as large as the cardinality of $\bP$. Conversely, in order to save nearly half\footnote{todo me}\todoReference{edges go to two} of that memory, one could store the value only once by implementing the control structures for detecting if an edge length has already been saved \todoReference{more details about this method in future work}, then when retrieving the values, one could search for the edge length required at the cost of compute time. In the next section, we choose to implement the first, speedier method.

%
%
\subsection{Parallel Variant of Calculate Edge Lengths}
\label{ch6sCELPssPVCEL}
In this section, we apply the knowledge gained from the analysis of Algorithm~\ref{alg:serialCalculateEdgeLengths} conducted in Section~\ref{ch6sCELPssASACEL}, in order to provide the pseudo code definitions with which one may implement a parallel variant of the function \textit{serialCalculateEdgeLengths}.

The mathematical pseudo code for the parallel algorithm for calculating all the edge lengths between each pair of adjacent points in the mesh is presented in Algorithm~\ref{alg:parallelCalculateEdgeLengths}. It requires the knowledge of and access to the number of available processors in the system $\rho$, the set of all points $\bP$, the family of sets of neighborhoods $\bN$, and the count of all neighbors in all neighborhoods $\hat{n}$, and produces as an output, the set of pre-calculated edge lengths $\bE$, and the global minimum edge length $\gelm$; both instrumental to the calculation of \Fors{t}. This algorithm has been split further into three subroutines in order to maximize efficiency by facilitating the balance of work loaded onto each processor in parallel.

\begin{algorithm}[ht]
	\algotitle{Parallel Algorithm for Calculating Edge Lengths}{paCEL.title}
	\DontPrintSemicolon
	\SetCommentSty{small}
	\SetKwFor{For}{for}{:}{}
	\SetKwProg{Sub}{Subroutine}{}{}
	\SetKwProg{Func}{Function}{}{}
	\SetKwInOut{Input}{Input}\SetKwInOut{Output}{Output}

	\Input{the number of available processors $\rho$, \\
		the set of all points $\bP$, \\
		the family of sets of neighborhoods $\bN$, \\
		the count of all neighbors in all neighborhoods $\hat{n}$}
	\Output{the set of pre-calculated edge lengths $\bE$, \\
		the global minimum edge length $\gelm$}

	\bigskip
\nl	\Func{parallelCalculateEdgeLengths($\rho$, $\bP$,\,$\bN$,\,$\hat{n}$)}{
\nl		$\sigma \leftarrow \hat{n}\mathbin{/}\rho$\;
\nl		$\bar{n} \leftarrow \hat{n}\mathbin{/}|\bP|$\;
\nl		\For{$\Pi \in \{1,\,\ldots,\,\left \lceil\rho/\bar{n}\right\rceil\}$}{
\nl			\ProgSty{$\sim$calculateLengths($\Pi$, $\sigma$, $\mu$, $\bP$,\,$\bN$)}\;
		}
\nl		\ProgSty{synchronizeThreads()}\;
	}

	\bigskip
\nl	\Sub{calculateLengths($\Pi$, $\sigma$, $\bP$,\,$\bN$)}{
\nl		$\check{\sigma} \leftarrow (\Pi-1)\,\sigma+1$\;
\nl		$\hat{\sigma} \leftarrow \Pi\,\sigma$\;
\nl		$\ProcSty{create}(\mu)$\;
\nl		\For{$\bp_v \in \{\bp_{\check{\sigma}},\ldots,\,\bp_{\hat{\sigma}}\}$}{
\nl			\For{$\bp_i \in \bN_v$}{
\nl				\ProgSty{$\sim$safeEdgeLengthCalculation($\mu$, $\bE$, $\gelm$, $\bp_v$, $\bp_i$)}\;
			}
		}
	}

	\bigskip
\nl	\Sub{safeEdgeLengthCalculation($\mu$, $\bE$, $\gelm$, $\bp_v$, $\bp_i$)}{
\nl		$\bE_{\sv{i}} \leftarrow |\bp_i - \bp_v|$\tcc*[r]{$\ellstar$, Eq:~\ref{eq:localMinimumEdgeLength}}\label{algPCELellstar}
\nl		\If(\tcc*[f]{heuristic only\footnotemark}){$\bE_{\sv{i}} < \gelm$}{\label{algPCELhcs}
\nl			$\ProcSty{lock}(\mu)$\;
\nl			$\gelm \leftarrow \min\left \{\gelm,\,\bE_{\sv{i}}\right \}$\label{algPCELgelm}\tcc*[r]{Eq:~\ref{eq:globalMinimumEdgeLength}}
\nl			$\ProcSty{unlock}(\mu)$\;
		}
	}
	\caption{Parallel algorithm for calculating all the edge lengths between each pair of adjacent points in the mesh\label{alg:parallelCalculateEdgeLengths}}
\end{algorithm}
\footnotetext{While it is true that we are attempting to avoid any unnecessary edge length calculations or mutex locks, the hidden message in this line is honestly just a happy accident.}
\todoBackground{lock/unlock/mutex}
\todoReword{$\bE_{\sv{i}}$ is a unique address, so no mutex required}
\todoBackground{future work, can calculate average neighborhood size in alg.1 or 4}

The initial function, \textit{parallelCalculateEdgeLengths}, requires all of the inputs listed in the previous paragraph, then calculates the stride with the problem size equating to the count of all neighbors in all neighborhoods $\hat{n}$. Next, the average size of all neighborhoods in $\bN$ is calculated as $\bar{n}$, and the single mutex $\mu$ is created to be used in protecting reads and writes to to $\gelm$. Then, the loop iterating over a portion of the total number of processors, scaled by the average neighborhood size, is begun, calling in each iteration, the subroutine \textit{calculateLengths}. Because each thread will be expected to spawn additional threads\footnote{the greater portion of the total thread count, overall.} for each pair of adjacent points found in each neighborhood in its designated stride of the work load, the greater portion of processors are held in reserve for those spawned threads to be able to run in parallel. Notice the ``ceiling'' operator $\left\lceil\cdot\right\rceil$, which ensures that at least instance of \textit{calculateLengths} will be executed. Finally, all the working threads must be synchronized before the set of pre-calculated edge lengths $\bE$, or the final value of global minimum edge length $\gelm$, may be used.

The subroutine \textit{calculateLengths} requires an index $\Pi$, which it uses along with the stride $sigma$, to calculate the lower and upper boundaries, $\check{\sigma}$ and $\hat{\sigma}$, for the stride of the total problem size which it should compute. Next, the function iterates over each point $\bp_v$ within its stride boundaries, as well as each neighbor $\bp_i$ in the current center point's neighborhood $\bN_v$, spawning new threads in each iteration to execute instances of the \textit{safeEdgeLengthCalculation} subroutine.

The \textit{safeEdgeLengthCalculation} subroutine requires the set of edge lengths $\bE$, which is being collaboratively populated by each other instance of the subroutine, as well as the single mutex $mu$, used to guard the final reading of and writing to the shared value of $\gelm$. Naturally, also required are the coordinates of the two points, between which the distance is being calculated. In line~\ref{algPCELellstar}, the first line of the function, the L2-norm of the difference between the two points is already calculated; this is the $\ellstar$ operation. There are two race conditions in line~\ref{algPCELgelm} which must be avoided, however it would be incredibly inefficient to have all $(\bar{n}-\rho)\mathbin{/}\bar{n}$ threads attempt to lock the same, single mutex $\mu$. The solution is the heuristic conditional statement in line~\ref{algPCELhcs}, testing the worthiness of incurring the cost of attempting to lock the shared mutex guarding the reads and writes to $\gelm$. We describe this conditional statement heuristic, in order to call attention to the fact that it may give an inaccurate result due to race conditions between the threads operating in parallel. \todoResearch{just how much time can be saved with this heuristic?}

Algorithm~\ref{alg:parallelCalculateEdgeLengths} produces as an output, the set of pre-calculated edge lengths $\bE$, and the global minimum edge length $\gelm$; both instrumental to the calculation of \Fors{t}, by calculating in parallel, each pair of adjacent points, in each neighborhood, comprising the family of sets of neighborhoods $\bN$. This algorithm has been split further into three subroutines in order to maximize efficiency by facilitating the balance of work loaded onto each processor in parallel.

In the next section, we will see how the pre-calculations of Algorithms~\ref{alg:parallelBuildNeighborhoods}, \ref{alg:parallelRecursiveCensusNeighborhoods}, and \ref{alg:parallelCalculateEdgeLengths} contribute, alongside the modifications to exploit the concurrency inherent to the filter, to the speed and scalability of the main procedure, defined in Algorithm~\ref{alg:parallelConvolveFilter}.

%
%
%
%
\section{Convolving the Filter in Parallel}
\label{ch6sCFP}
Finally, in this section, we examine the serial Algorithm~\ref{alg:serialConvolveFilter}, which includes the principle loop to convolve \Fors{t}. As a prerequisite, one must have successfully executed the previous three parallel algorithms, as detailed in Sections~\ref{ch6sBNP} and~\ref{ch6sCELP}, in order to generate the input parameters which will be used in the parallel variant of this algorithm. First we will analyze the serial algorithm through the lens of parallel programing, to be followed by defining the strategy behind the design of the parallel variant, then finally providing the pseudo code definitions to implement Algorithm~\ref{alg:parallelCalculateEdgeLengths}.

%
%
\subsection{Analysis of Serial Algorithm~\ref{alg:serialConvolveFilter}: Convolve Filter}
\label{ch6sCFPssASACF}
In this section, we analyze through the lens of parallel programing, the~\nameref{saCF.title} as presented in Section~\ref{ch5sCF}. The strategy we employ will match that of the previous three algorithms, which involves the reading of the serial algorithm line by line, building a dependency graph, and as opportunities for exploiting parallelism are discovered, devising a strategy\todoReword{strategy twice is bad} for maximizing efficiency using parallel techniques.

Starting in line~\ref{scf1} of Algorithm~\ref{alg:serialConvolveFilter}, we read the declaration of the function \textit{serialConvolveFilter} requiring as inputs: the set of points $\bP$, the set of function values $\bF$, the user-defined number of convolutions to perform $\tau$, as well as the information generated in the previous three algorithms, the family of sets of neighborhoods $\bN$, the set of pre-calculated edge lengths $\bE$, and the global minimum edge length $\gelm$. Then, in the first line of the function \textit{serialConvolveFilter}, we encounter a loop which runs for the user-defined number of convolutions $\tau$. In order to determine if one can parallelize the entire loop, once must closely analyze each operation within the loop block; searching for loop-carried control or data dependencies.

So, skipping ahead to the last line of the loop block, line~\ref{algSCFlastLine}, we read that the set of function values $\bF$ is replaced by the newly computed set of \wmfv{s} $\bF'$, before starting the next convolution. This models exactly the definition of a loop-carried data dependency, therefore it would be impossible to efficiently calculate this loop entirely\footnote{What may be possible, would be to pipeline calculations for selected points between convolutions, since each calculation only actually requires the new function values of its neighbors, not the entire set. However, given the effort required to compute a single convolution of \fors{t} on typical acquired \tdd{}, it is unlikely that any additional speedup will be realized using such a technique, computing with the GPGPUs commercially available today.} in parallel.

In the second and third lines of function \textit{serialConvolveFilter}, we encounter two more loops iterating over each point in $\bP$, and then each neighborhood $\bN_v$ associated with those points. We have seen this pattern before in Section~\ref{ch6sCELPssASACEL}; together, these two loops iterate over the non-deterministic membership of the family of sets of neighborhoods $\bN$, and indeed can be unrolled and computed in parallel by utilizing the total count of neighbors and the average size of all neighborhoods, barring any occurrences of loop-carried dependencies within the loop block.

Figure~\ref{fig:sacfDataDependencies} illustrates the data dependencies inherent to each iteration of the pair of loops over the points $\bp_v$ in $\bP$, and then $\bp_i$ in $\bN_v$, as found in lines~\ref{scf1}~-~\ref{algSCFlastLine} of Algorithm~\ref{alg:serialConvolveFilter}.

\begin{figure}[ht]
	\includestandalone[width=1.0\textwidth]{figures/tikz/sacfDataDependencies}
	{\caption[Data Dependencies in Serial Algorithm~\ref{alg:serialConvolveFilter}: Calculate Edge Lengths]{An illustration of the data dependencies found in Algorithm~\ref{alg:serialConvolveFilter}. Square nodes indicate a values stored in memory, oval nodes indicate operations. Single arrows indicate reading values, double arrows indicate writing values to memory. Drawn in teal color are static memory, and the values and operations which are independent and free of data dependencies. Drawn in sand color are mutable memory, the values stored there, and the operations which rely on them. These are independent, however may be modified during processing. Drawn in coral color, are volatile memory, the values stored there, which may be modified at anytime during a procedure, and the operations which rely on those volatile values.}\label{fig:sacfDataDependencies}}
\end{figure}

The first several operations exclusively require values stored in static memory, a fact which exposes an opportunity for exploiting the independence of these operations by computing them in parallel. As denoted by the teal colored lines and nodes, the current point $\bp_v$ is read from $\bP$ in static memory. Next, that point is referenced to read the corresponding neighborhood $\bN_v$ from $\bN$ in static memory. Then the points $\bp_i$ and $\bp_{\sipo}$ are determined, and combined with $\bp_v$ in order to reference the precomputed edge lengths $\bE_a$, $\bE_b$, and $\bE_c$, stored in $\bE$, also in static memory. The three edge lengths are then used in the operation to calculate $\alpha$, which is subsequently used to calculate $\beta$, then combined with $\gelm$ from static memory to calculate $A$ and $\check{\ell}$. Next, $\beta$ combines with $\gelm$ in an operation to calculate $\zeta$, which is then used with the edge lengths $\bE_b$ and $\bE_c$ in order to compute $\tilde{\ell}_j$ and $\tilde{\ell}_{\sjpo}$.

Then, drawn in sand color to highlight the fact that the values stored in mutable memory may change between iterations, we see that the function value in $\bN_v$, of each point already in consideration, is read from mutable memory as the values $f_0$, $f_j$, and $f_{\sjpo}$. Next, $\tilde{\ell}_j$ and $\tilde{\ell}_{\sjpo}$ combine with $f_j$, and $f_{\sjpo}$ in two operations which separately calculate the values $f'_j$, and $f'_{\sjpo}$, which are then combined with $\check{\ell}$ and $f_0$ to eventually calculate $\check{f}$.

Next, drawn in coral color to highlight the fact that the correctness of the values stored in volatile memory cannot be guaranteed without the protection of a locking mechanism, the current values of $\tilde{f}_v$ and $\tilde{A}_v$ are read from volatile memory, so that $\tilde{f}_v$ may combine with $A$ and $\check{f}$ in order to accumulate the value of $\tilde{f}_v$, while $\tilde{A}_v$ combines with $A$ in order to accumulate the value of $\tilde{A}_v$. Once $\tilde{f}_v$ and $\tilde{A}_v$ have totally accumulated, and their threads have been synchronized, drawn in sand color, we see that $\tilde{f}_v$ and $\tilde{A}_v$ combine to finally calculate the value $f'_v$, which is stored in mutable memory to be utilized in subsequent computations.

Having now considered the dependency graph in Figure~\ref{fig:sacfDataDependencies}, we shall continue analyzing, line by line, the Serial Algorithm~\ref{alg:serialConvolveFilter}, returning to the beginning of the loop block of the inner most nested loop.

The Lines~\ref{algSCFalpha}~-~\ref{algSCFcheckf} only generate new terms which are unique to the current iteration, using only original values which are not changed by any other thread, therefore every operation in the block is a prime candidate for exploiting concurrency in the loop. Line~\ref{algSCFjloop} starts a new loop over just two values, and while it would be possible to compute both halves of the values the loop in parallel, at this point in the computation, we would expect all processors to already be in use calculating the other iterations, so dedicating the resources to spawn new threads here would likely net a loss in efficiency, so we will simply unroll the loop and compute the four new terms in serial, per thread.

Lines~\ref{algSCFtildef}~and~\ref{algSCFtildeA} both accumulate values in shared memory which will be accessed in parallel, on average $\bar{n}$ times, setting up race conditions and threatening the correctness of the algorithm. Because of the limited number of possible collisions, we will treat these two lines as a single critical section, protecting each pair with a shared mutex as we did for the paired union operations in Algorithm~\ref{alg:parallelBuildNeighborhoods}. However, in contrast, these mutexes will have a limited scope, not being shared globally, but only shared within a local neighborhood.

In line~\ref{algSCFfprimev}, the values $\tilde{f_v}$ and $\tilde{A_v}$, having been accumulated for each neighbor in $\bN_V$, are used to calculate the \wmfv{} for the entire geodesic disc $f'_v$, but before their values are read, all threads working on that neighborhood must be synchronized to ensure the correct values are used. As thread synchronization is quite expensive to computation time\todoReference{syncing is expensive}, it is imperative that only those threads working on the neighborhood are synchronized, and not all threads, as is necessary in most other parts of this algorithm, including between the next two lines.

In lines~\ref{algSCF2ndlastLine}~and~\ref{algSCFlastLine}, all the values of $f'_v$ from each neighborhood in $\bN$ are collected into the new set $\bF'$, then either returned as the final result of the filter's convolutions, or used as the scalar field of function values in the next iteration of the filter. Storing the values in the new set may be done in parallel, because each value will remain unique in the set, however, before the new set may be used in either of those two ways, all threads of execution must have time to synchronize in order to ensure that all values have had time to populate the set, thus avoiding the use of incorrect values.

Now that we have analyzed, the serial Algorithm~\ref{alg:serialConvolveFilter}, and discovered opportunities where exploiting the loop-level concurrency will benefit the efficiency of \fors{t}, let us not hesitate to create the parallel variant.

%
%
\subsection{Parallel Variant of Convolve Filter}
\label{ch6sCFPssPRCN}
In this section, we apply the knowledge gained from the analysis of Algorithm~\ref{alg:serialConvolveFilter} conducted in Section~\ref{ch6sCFPssASACF}, in order to provide the pseudo code definitions with which one may implement a parallel variant of the function \textit{serialConvolveFilter}.

Algorithm~\ref{alg:parallelConvolveFilter} is split into three parts, the main function \textit{parallelConvolveFilter}, and two subroutines \textit{safeAccumulateGeoDiscMean} and \textit{safeAccumulateSectorMean}, in order to facilitate the delegation of portions of the total work load to separate threads of execution, while protecting critical sections, and avoiding data and control dependencies, so that the threads can be safely executed in parallel. The main function \textit{parallelConvolveFilter} requires several inputs, as did its serial counterpart, which include: the set of points $\bP$, the set of function values $\bF$, the user-defined number of convolutions to perform $\tau$, as well as the information generated in the previous three algorithms, the family of sets of neighborhoods $\bN$, the set of pre-calculated edge lengths $\bE$, and the global minimum edge length $\gelm$. In addition, it also requires the number of available processors $\rho$, and the count of all neighbors in all neighborhoods $\hat{n}$, specifically for this parallel variant of the algorithm.

In line~\ref{algPCFtauloop}, the first line of the function, the loop which iterates for the user-defined number of times $\tau$ begins. Initially, the stride $\sigma$ is calculated with the work load computed as one for each point defining a geodesic disc $\bO$, which is equal to the cardinality of points in the mesh $|\bP|$, plus one for each neighboring point in every neighborhood $\hat{n}$. Next, another loop is begun, which iterates over a portion of the available processors in the system, as we did in Algorithm~\ref{alg:parallelCalculateEdgeLengths}, reserving resources for the rest of the threads that must be spawned in order to compute in parallel, the \wmfv{} over an entire neighborhood $f'_v$. In each iteration of this loop, a new thread is spawned to execute i  parallel the subroutine \textit{safeAccumulateGeoDiscMean}. At the conclusion of the loop block, all threads are synchronized before either the new scalar field of \wmfv{s} $\bF'$ is used as the function values upon which the next convolution of the filter is based, or the convolutions end and $\bF'$ is returned as the final result.

The subroutine \textit{safeAccumulateGeoDiscMean}, whose definition begins in line~\ref{algPCFsagdmCall}, is called for each processor not reserved for sector-wise computations, being passing to it as parameters: a process index $\Pi$, the stride length $\sigma$, the new scalar field which is to be populated with calculated \wmfv{s} from each neighborhood $\bF'$, and the other general information defining and describing the mesh, $\bP$, $\bN$, $\bE$, $\gelm$, $\bF$. This subroutine begins by using its designated index $\Pi$, to compute the lower and upper boundaries, $\check{\sigma}$ and $\hat{\sigma}$, for the stride of the total problem size on which it should compute. Next, the subroutine iterates over each point $\bp_v$ within its stride boundaries, creating a new set of mutexes $\fM$, containing one $\mu$ per neighbor in currently neighborhood $\bN_v$, before beginning another loop. This nested loop iterates over each neighbor $\bp_i$ in the current neighborhood $\bN_v$, and spawns new threads to execute instances of the \textit{safeAccumulateSectorMean} subroutine. After the loop over each point in $\bN_v$ is complete, the threads which were used to calculate the \wmfv{s} for each circle sector are synchronized, before their values are used in the calculation of $f'_v$, the \wmfv{} at point $\bp_v$, which is collected in the set $\bF'$. While $\bF'$ is a set shared among all the threads, the assignment of $f'_v$ does not require the protection of a locking mechanism against race conditions, because during any given convolution of the filter, only one single thread will ever compute and attempt to store the value of $f'_v$ at the location $\bF'_v$.

In line~\ref{algPCFsasmCall}, a new thread is spawned to execute an instance of the function \textit{safeAccumulateSectorMean} for each sector in the current neighborhood. The first nine lines of \textit{safeAccumulateSectorMean} culminate in the computation of $\check{f}$, the \wmfv{}, for the current circle sector being processed. As seen in Figure~\ref{fig:sacfDataDependencies}, because each of these operations are free of data and control dependence, besides the dependence of the results of the other operations in this block, these nine instructions may all be computed in serial by individual threads, without requiring the expensive protection of a locking mechanism.

Furthermore, because the goal of this block is to accumulate these \wmfv{s} in line~\ref{pcf.tildefv}, as well as the area for each sector in line~\ref{pcf.tildeAv}, so that both may be used in the calculation of the weighted mean over the entire geodesic disc in line~\ref{pcf.Fprimev}, we chose to guard the shared memory values $\tilde{f}_v$ and $\tilde{A}_v$ using mutexes, risking on average a small number of collisions, only $\bar{n}$ per neighborhood. An alternative solution, which we choose not to implement here because it would cost $2\cdot\hat{n}$ more floating point values worth of memory, is to save each intermediate value separately, then perform the averaging calculation on those values after having synchronized all the threads.

\begin{algorithm}[ht]
	\algotitle{Parallel Algorithm for Convolving the Filter}{paCF.title}
	\DontPrintSemicolon
	\SetCommentSty{small}
	\SetKwFor{For}{for}{:}{}
	\SetKwProg{Func}{Function}{}{}
	\SetKwProg{Sub}{Subroutine}{}{}
	\SetKwInOut{Input}{Input}\SetKwInOut{Output}{Output}

	\Input{the number of available processors $\rho$, \\
		the set of all points $\bP$, \\
		the family of sets of neighborhoods $\bN$, \\
		the count of all neighbors in all neighborhoods $\hat{n}$, \\
		the set of pre-calculated edge lengths $\bE$, \\
		the global minimum edge length $\gelm$, \\
		the set of function values $\bF$, \\
		the user-defined number of convolutions $\tau$}
	\Output{the set of one-ring \wmfv{s} $\bF'$}

	\bigskip
	\linespread{1}\selectfont
\nl	\Func{parallelConvolveFilter($\rho$, $\bP$, $\bN$, $\hat{n}$, $\bE$, $\gelm$, $\bF$, $\tau$)}{
\nl		\For{$t\leftarrow 1\;\KwTo\;\tau$}{\label{algPCFtauloop}
\nl			$\sigma \leftarrow (|\bP|+\hat{n})\mathbin{/}\rho$\;
\nl			$\bar{n} \leftarrow \tilde{n}\mathbin{/}|\bP|$\;
\nl			\For{$\Pi \in \{1,\ldots,\left \lceil\rho/\bar{n}\right\rceil\}$}{
\nl				\ProgSty{$\sim$safeAccumulatesGeoDiscMean($\Pi$, $\sigma$, $\bP$, $\bN$, $\bE$, $\gelm$, $\bF$, $\bF'$)}\;\label{algPCFsagdmCall}
			}
\nl			\ProgSty{synchronizeThreads()}\;
\nl 		\If(\tcc*[f]{smooth newest values every convolution}){$t < \tau$}{$\bF \leftarrow \bF'$}
		}
\nl 	\KwRet $\bF'$\;
	}

	\bigskip
\nl	\Sub{safeAccumulatesGeoDiscMean($\Pi$, $\sigma$, $\bP$, $\bN$, $\bE$, $\gelm$, $\bF$, $\bF'$)}{
\nl		$\check{\sigma} \leftarrow (\Pi-1)\,\sigma+1$\;
\nl		$\hat{\sigma} \leftarrow \Pi\,\sigma$\;
\nl		\For{$\bp_v \in \{\bp_{\check{\sigma}},\ldots,\,\bp_{\hat{\sigma}}\}$}{
\nl			$\fM \leftarrow \left \{\mu_1,\dots,\mu_{|\bN_v|}\right \}$\;
\nl			\For{$\bp_i \in \bN_v$}{
				\smallskip
\nl				\ProgSty{$\sim$safeAccumulateSectorMean($\fM$, $\bE$, $\gelm$, $\bp_v$, $\bp_i$, $\bF$, $\tilde{f}_v$, $\tilde{A}_v$)}\;\label{algPCFsasmCall}
			}
\nl			\ProgSty{synchronizeThreads(\{$\check{\sigma}$, \ldots, $\hat{\sigma}$\})}\;
\nl			$\bF'_v \leftarrow \tilde{f}_v\mathbin{/}\tilde{A}_v$\tcc*[r]{as $f'_v$, Eq:~\ref{eq:meanFuncValAtPv}}\label{pcf.Fprimev}
		}
	}

	\bigskip
\nl	\Sub{safeAccumulateSectorMean($\fM$, $\bE$, $\gelm$, $\bp_v$, $\bp_i$, $\bF$, $\tilde{f}_v$, $\tilde{A}_v$)}{
		\linespread{1.5}\selectfont
\nl		$\kern-0.5pt\alpha \leftarrow cos^{-1}$
		\begin{Large}
			$\kern-6pt\left (\frac{\bE_c^2\,+\,\bE_b^2\,-\,\bE_a^2}{2\,\cdot\,\bE_c\,\cdot\,\bE_b}\right )$\tcc*[r]{Eq:~\ref{eq:alphaFromEdgeLengths}}
		\end{Large}
		\linespread{1.2}\selectfont
\nl		$\kern0.00pt\beta \leftarrow (\pi - \alpha)\mathbin{/}2$\tcc*[r]{Eq:~\ref{eq:betaFromHalfAlpha}}
\nl		$\kern-1.5ptA \leftarrow \Big(\gelm\,\Big)^2\kern-4pt\cdot\alpha\mathbin{/}2$\tcc*[r]{Eq:~\ref{eq:circularSectorArea}}
\nl		$\kern1.00pt\check{\ell} \leftarrow \big(4\cdot\gelm\cdot\sin(\alpha\mathbin{/}2)\big)\mathbin{/}3\,\alpha$\tcc*[r]{Eq:~\ref{eq:distToCoG}}
\nl		$\kern1.00pt\zeta \leftarrow \gelm\mathbin{/}\sin(\beta)$\tcc*[r]{Eq:~\ref{eq:zeta}}
\nl		\For{$j \in {1,2}$}{
\nl			$\tilde{\ell}_j \leftarrow \zeta\mathbin{/}\bE_j$\tcc*[r]{Eq:~\ref{eq:distanceIForInterpolationalgSCFlastLine},~\ref{eq:distanceIp1ForInterpolation}}
\nl			$f'_j \leftarrow f_0\cdot(1 - \tilde{\ell}_j) + f_j\cdot\tilde{\ell}_j$\tcc*[r]{Eq:~\ref{eq:interpolatedFi},~\ref{eq:interpolatedFip1}}
		}
\nl		$\check{f} \leftarrow f_0\cdot(1 - \check{\ell}) + \big((f'_1 + f'_2)\cdot\check{\ell}\big)\mathbin{/}2$\tcc*[r]{Eq:~\ref{eq:weightedMeanAtCoGatSector}}
		\linespread{1.0}\selectfont
\nl		$\ProcSty{lock}(\mu)$\;
\nl		$\kern2.0pt\tilde{f}_v \leftarrow \tilde{f}_v + A\cdot\check{f}$\tcc*[r]{Eq:~\ref{eq:meanFuncValAtPv}}\label{pcf.tildefv}
\nl		$\kern0.0pt\tilde{A}_v \leftarrow \tilde{A}_v + A$\tcc*[r]{Eq:~\ref{eq:meanFuncValAtPv}}\label{pcf.tildeAv}
\nl		$\ProcSty{lock}(\mu)$\;
	}

	\caption{Parallel algorithm for convolving \Fors{t}\label{alg:parallelConvolveFilter}}
\end{algorithm}

The final result is that the \wmfv{s} $\tilde{f'}_v$ and areas $\tilde{A}_v$ of each circle sector are calculated in parallel, then those values are averaged, also in parallel, over the total area of the geodesic disc centered at each $\bp_v$, and collected in the set $\bF'$, completing a single convolution of \Fors{t}.

%
%
%
%
\section{Summary}
In this chapter, we explored the serial version of this as-yet-unpublished algorithm, in order to negotiate any instances of control or data dependency, and were able to discover several manifestations of independent procedures worthy of exploiting with parallel processing, which when implemented on a system capable of parallel computation, significantly improves the performance of \Fors{t}. The parallel algorithm presented in this chapter has three main parts, each split further into subroutines, to facilitate the simultaneous execution of blocks of instructions by threads in parallel. This algorithm was developed by first by analyzing the corresponding serial algorithm through the lens of parallel programing, defining the strategy behind the design of the parallel variant with the aid of data dependency diagrams, then finally providing the pseudo code definitions to implement each part of the algorithm.

