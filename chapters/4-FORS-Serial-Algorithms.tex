\chapter{Fast One-Ring Smoothing: Serial Algorithms}
\label{ch4}
In the previous chapter we presented the mathematical grounding for an improved version of \Fors{t} as it is currently implemented within the GigaMesh framework. Now in this chapter, we will use mathematical pseudo-code to combine all of the equations from Chapter~\ref{ch3}, Equations~\ref{eq:localMinimumEdgeLength} through~\ref{eq:meanFuncValAtPv}, into a serial algorithm, with the goal of preparing the basis from which we will subsequently design the parallel algorithm presented in Chapter~\ref{ch6}.

%
%
\section{An Algorithm in Three Parts}
\label{ch4sATP}
The serial algorithm defined in this chapter requires that the membership of all one-ring neighborhoods first be discovered before one is ever able to begin convolving the filter, because the total efficiency of the algorithm relies on iterating over the family of sets of neighborhoods $\bN$, as will be made clear in Sections~\ref{ch4sCEL}, and~\ref{ch4sCF}, while the underlying data structure of \tdd{} typically only stores a mesh $\bM$ as the superset of the two separate sets, $\bP$ of points, and $\bT$ of triangular faces, completely ignorant of the adjacency between faces or even the concept of neighborhoods.

Also critical to the total efficiency of the filter, is the pre-calculation of the edge lengths between all pairs of adjacent points, whose magnitudes never change, regardless of the number of convolutions being applied. This importance stems from how all edge lengths are used in computation during the loop for convolving the filter, which we will refer to as ``the \gls{principleLoop}'' going forward, at least five times per convolution, and tens times per convolution for the majority of points, which are non-border pairs, whose relationships are duplicated in adjacent neighborhoods.\todoReference{appendix on mesh face to point ratio goto2}

Therefore, in order to maximize the efficiency of \Fors{t}, it is crucial to split the algorithm into three distinct parts: first building neighborhoods, then calculating edge lengths, and finally convolving the filter, while storing the results of the first two parts, so that they may be used during the iterative convolutions of the third part; the result being a massive increase in efficiency by greatly reducing the number of operations-per-convolution required by the filter.
\todoBackground{add convolution and convolve to background}
\todoBackground{memory vs speed cost compromise}

%
%
%
%
\section{Building Neighborhoods}
\label{ch4sBN}
Before convolving \fors{t}, one must initially discover all the points in the mesh $\bp_i$ which comprise each neighborhood $\bN_v$, then store those connections in the family of sets $\bN$. That is the purpose of Algorithm~\ref{alg:serialBuildNeighborhoods}. Despite the fact that building $\bN$ outside of the principle loop adds an additional $6\cdot|\bT|$ operations to the total, instead, Algorithm~\ref{alg:serialCalculateEdgeLengths} becomes able to exploit the explicit connections stored in $\bN$, in order to vastly reduce its complexity from $|\bP|^{|\bT|}$, down to only $|\bP|^{\bar{n}}$, where $\bar{n}$ is the average size of all neighborhoods, which with acquired \tdd{}, will typically evaluate to approximately 6.\todoReference{Appendix on border vs nonborder ,and beighborhood size} Also, as we will see in Section~\ref{ch4sCF}, when $\tau$ is the user-defined number of convolutions to perform, the complexity of the principle loop can be meaningfully reduced to only $\tau^{|\bP|^{2\cdot\bar{n}}}$, down from the $\tau^{|\bP|^{(2\cdot|\bT|)}}$ that would have been necessary, had the neighborhoods not already been discovered and the procedure been otherwise required to discover the members of $\bN_v$ in each convolution.%
\nomenclature[na]{$\tau$}{the user-defined number of convolutions to perform}%
\todoBackground{complexity, big O notation}

Figure~\ref{fig:unionsOfSimpleBuildNeighborhoods} describes a very simple mesh consisting of just two faces and four points, similar to what is illustrated in Figure~\ref{fig:triangularFaces}. The arrows represent the union operation between a point and a neighborhood, and are colored to match the face from where the point had come. It should be noted that the two pairs of arrows pointing from $\bp_2$ to $\bN_3$, and $\bp_3$ to $\bN_2$, indicate that both of these union operations occur twice, originating one each from both faces, but because of the uniqueness property of sets, the duplicated operations are wholly inconsequential to the the final membership of either neighborhood.

\begin{figure}[ht]
	\includestandalone[width=\textwidth]{figures/tikz/unionsOfSimpleBuildNeighborhoods}
	{\caption[unionsOfSimpleBuildNeighborhoods]{A very simple mesh consisting of just two faces and four points, similar to what is illustrated in Figure~\ref{fig:triangularFaces}. The face $\bt_1$ is in sand color, and $\bt_2$ is in coral color. The arrows represent the union operation between a point into a neighborhood, and are colored to match the face from where the point had come. The two pairs of arrows pointing from $\bp_2$ to $\bN_3$ and $\bp_3$ to $\bN_2$ are colored in teal to highlight the fact that these union operations occur twice.
}\label{fig:unionsOfSimpleBuildNeighborhoods}}
\end{figure}

Also notice in Figure~\ref{fig:serialBuildNeighborhoods}, that the union operation is performed exactly twice per point per face; that is, once each between the neighborhood of the center point and a neighboring point, for a total of six times per face. For example: the triangular face $\bt_2$ contains the point $\bp_4$, therefore the union operation is performed on $\bN_4$ at least two times, once for each of the other members of point $\bt_2$: $\bp_2$, and $\bp_3$. This realization will influence the design of the parallel variant of the algorithm, as described in Section~\ref{ch6sBNPssABN}.


In Algorithm~\ref{alg:serialBuildNeighborhoods}, the function \textit{serialBuildNeighborhoods} describes iterating over every triangular face $\bt$ in $\bT$, then for each face's three corner points, perform the union operation between the neighborhood centered at the point, and the other two points which are adjacent to the central point. After processing every face, the result becomes a fully populated family of sets $\bN$, storing the connections between every neighbor of every neighborhood in the mesh $\bM$.

\begin{algorithm}[ht]
	\algotitle{Serial Algorithm for Building Neighborhoods}{saBN.title}
	\DontPrintSemicolon
	\SetCommentSty{small}
	\SetKwFor{For}{for}{:}{}
	\SetKwProg{Func}{Function}{}{}
	\SetKwInOut{Input}{Input}\SetKwInOut{Output}{Output}

	\Input{the set of all triangular faces $\bT$}
	\Output{the family of sets of discovered neighborhoods $\bN$}

	\bigskip
\nl	\Func{serialBuildNeighborhoods($\bT$)}{\label{sbn1}
\nl		\For(\tcc*[f]{$\bt = \left \{\bp_a, \bp_b, \bp_c\right \}$}){$\bt \in \bT$}{\label{sbn2}
\nl			\ProgSty{union($\bN$, $\bp_a$, $\bp_b$, $\bp_c$)}\;\label{sbn3}
\nl			\ProgSty{union($\bN$, $\bp_b$, $\bp_a$, $\bp_c$)}\;\label{sbn4}
\nl			\ProgSty{union($\bN$, $\bp_c$, $\bp_a$, $\bp_b$)}\;\label{sbn5}
		}
	}

	\bigskip
\nl	\Func{union($\bN$, $a$, $b$, $c$)}{\label{sbn7}
\nl		$\bN_a \leftarrow \bN_a \cup \{b,\,c\}$\;\label{sbn8}
	}
	\caption{Serial algorithm for building the family of sets $\bN$, from all discovered members of each neighborhood in the mesh\label{alg:serialBuildNeighborhoods}}
\end{algorithm}%

The function \textit{union} is defined separately in Algorithm~\ref{alg:serialBuildNeighborhoods} for two reasons: the first reason is that when separated, it becomes very clear that the union operation behaves similarly for all three permutations\footnote{In general, a set of three numbers has six permutations, however, here only the first index is important, and the order of the second two parameters are arbitrary due to the commutative property of the union operation, as shown in Equation~\ref{eq:ascAndComPropertiesOfUnions}, resulting in only 3 distinct possibilities} of corners, and indeed it is only the order of the indices which changes; the second reason is that this signature will match more closely that of its parallel variant, to be defined in Algorithm~\ref{alg:parallelBuildNeighborhoods}, which will require the union operation to remain separate.

%
%
%
%
\section{Calculating Edge Lengths}
\label{ch4sCEL}
Having now built in the previous section the family of sets of neighborhoods $\bN$, we can advance to the next step, Algorithm~\ref{alg:parallelCalculateEdgeLengths}, which iterates over each pair of neighbors comprising $\bN$, with the goal of building a set of pre-calculated edge lengths $\bE$, as well as determining the global minimum edge length $\gelm$; both being essential parameters of Algorithm~\ref{alg:serialConvolveFilter}.

As shown in Equation~\ref{eq:defineEdgeLengthPoint}, the calculation of an edge's length requires taking the L2-norm of the difference between two points, which itself involves using the square root operation. In modern software, the square root operation is performed by computing ``Newton's Iteration'', or ``Newton's method''\todoCitation{software uses Newton's iteration for sqrt}, which is essentially multiple iterations of the so-called, ``recurrence equation''.~\cite{Weisstein19b}\todoCitation{Newton's iteration uses recurrence equation} The impact for \fors{t} is that the computation of a square root typically\todoReword{add footnote with desc and citation}\todoCitation{how slow is Newton's iteration compared to others} takes many more compute cycles than  any other binary or unary operation, thus taking more time to complete overall. In fact, because of the slowness of the square root operation, computing the L2-norm in order to calculate an edge's length is empirically the most costly operation performed by the filter\todoResearch{qualify, do experiment to prove how slow sqrt is, maybe make appendix entry about what it is and why it is so slow}. Therefore, it is imperative that we pay special attention to avoid unnecessary instructions to calculate an edge's length. For that reason, we define the symbol $\ellstar$ to represent the calculation of an edge's length using ``Newton's Iteration'', so that we may draw focus to its importance while remaining concise.%
\nomenclature[oa]{$\ellstar$}{the procedure of calculating an edge's length using ``Newton's Iteration'', the most costly operation in the Fast One-Ring filter, due to use of $\sqrt{(\cdot)}$}

In Algorithm~\ref{alg:serialCalculateEdgeLengths}, the function \textit{serialCalculateEdgeLengths} iterates over a set of nested loops which considers each neighbor $\bp_i$ of each neighborhood $\bN_v$, in order to calculate the edge lengths between the center point $\bp_v$ and its neighboring points $\bp_i$. The result is then stored in the set $\bE$ using the pair of indices $v, i$ to reflect the structure of the family of sets of neighborhoods $\bN$. Finally, in each iteration, the value of the global minimum edge length $\gelm$ is updated to become the minimum between $\bE_{\sv{i}}$ and the current $\gelm$, ensuring that at the conclusion of the procedure, that no other edge length in $\bE$ will be shorter than $\gelm$.

\begin{algorithm}[ht]
	\algotitle{Serial Algorithm for Calculating Edge Lengths}{saCEL.title}
	\DontPrintSemicolon
	\SetCommentSty{small}
	\SetKwFor{For}{for}{:}{}
	\SetKwProg{Func}{Function}{}{}
	\SetKwInOut{Input}{Input}\SetKwInOut{Output}{Output}

	\Input{the set of all points $\bP$, \\
		the family of sets of discovered neighborhoods $\bN$}
	\Output{the set of pre-calculated edge lengths $\bE$, \\
		the global minimum edge length $\gelm$}

	\bigskip
\nl	\Func{serialCalculateEdgeLengths($\bP$, $\bN$)}{\label{scel1}
\nl		\For{$\bp_v \in \bP$}{\label{scel2}
\nl			\For{$\bp_i \in \bN_v$}{\label{scel3}
				\linespread{1.5}\selectfont
\nl				$\bE_{\sv{i}} \leftarrow |\bp_i - \bp_v|$\tcc*[r]{This is $\ellstar$ as in Eq:~\ref{eq:localMinimumEdgeLength}}\label{scel4}
\nl				$\gelm \leftarrow \min\left \{\gelm,\,\bE_{\sv{i}}\right \}$\tcc*[r]{Eq:~\ref{eq:globalMinimumEdgeLength}}\label{scel5}
			}
		}
	}
	\caption{Serial algorithm for calculating all the edge lengths between each pair of adjacent points in the mesh\label{alg:serialCalculateEdgeLengths}}
\end{algorithm}

Given the substantial impact of computing ${\ellstar}$, and the enormous number\footnote{``enormous'' $\approx2\cdot\left (|\bP|^{5\,\bar{n}}\right )$ as the non-border to border edge lengths converges to 2 with increasing mesh density, described in detail in the appendix}\todoResearch{update the 2 once the real value is known} of times an edge length is required in the computation of the \wmfv{s} per convolution of the filter, pre-calculating the set all of edge lengths $\bE$ outside of the principle loop becomes critical to the overall efficiency of the algorithm, despite the fact that it requires ${\ellstar}$ to be calculated and stored $|\bP|^{\bar{n}}$ times. It is of paramount importance for two reasons: the first reason is that without pre-calcuating every edge length, it would be otherwise impossible to calculate the global minimum edge length $\gelm$, which is used $|\bP|^{4\,\bar{n}}$ times in every convolution of the filter; the second reason is that by recording the results of every $\ellstar$ calculation in the set $\bE$, we are then able to completely exclude any further calculations of $\ellstar$ from the principle loop, and as can be seen in Algorithm~\ref{alg:serialConvolveFilter}, that reduces the total count of $\ellstar$ calculations performed by the filter from the $\tau^{|\bP|^{3\,\bar{n}}}$ had the procedure been required to calculate an edge length each time it was used during computation, down to only the initial $1\cdot |\bP|^{\bar{n}}$ pre-calculations; having now become completely independent of $\tau$ and significantly more efficient overall.%
\nomenclature[ob]{$\bE$}{a set of pre-calculated edge lengths}%
\todoBackground{memory vs speed cost compromise}
\todoBackground{border vs non-border edge lengths}

%
%
%
%
\section{Convolving the Filter}
\label{ch4sCF}
After having discussed in Section~\ref{ch4sBN}, the discovery and subsequent construction of the set of every one-ring neighborhood $\bN$, then in Section~\ref{ch4sCEL}, the set of pre-calculations of every edge length $\bE$, in this section, we present the third and final part of the serial algorithm for convolving \Fors{t}, which describes the remaining steps required to convolve the filter over a mesh.

In Algorithm~\ref{alg:serialConvolveFilter}, the function \textit{convolveFilter} outlines how the convolutions of \fors{t} are performed for a user-defined number of times $\tau$, by convolving over each and every point $\bp_v$ in the set $\bP$. Then at each point, each neighboring point $\bp_i$ in the one-ring neighborhood $\bN_v$ is examined, in order to calculate the \wmfv{} $\check{f}$, as described in detail in Sections~\ref{ch3sIA} -~\ref{ch3sWM}, at the center of gravity of the circle sector defined by $\bp_v$ and $\bp_i$. Next, the \wmfv{} $f'_v$ is evaluated as the average of all the \wmfv{s} from each circle sector comprising the geodesic disc $\bO_v$, and is stored  in set $\bF'$, before the filter finally progresses to the next point in the mesh and its corresponding one-ring neighborhood.  Afterwards, one may efficiently convolve the filter, for as many number of convolutions as required to achieve the desired smoothing effect.

\begin{algorithm}[ht]
	\algotitle{Serial Algorithm for Convolving the Filter}{saCF.title}
	\DontPrintSemicolon
	\SetCommentSty{small}
	\SetKwFor{For}{for}{:}{}
	\SetKwProg{Func}{Function}{}{}
	\SetKwInOut{Input}{Input}\SetKwInOut{Output}{Output}

	\Input{the set of all points $\bP$, \\
		the family of sets of discovered neighborhoods $\bN$, \\
		the set of pre-calculated edge lengths $\bE$, \\
		the global minimum edge length $\gelm$, \\
		the set of function values $\bF$, \\
		the user-defined number of convolutions $\tau$}
	\Output{the set of one-ring \wmfv{s} $\bF'$}

	\bigskip
	\linespread{1}\selectfont
\nl	\Func{serialConvolveFilter($\bP$, $\bN$, $\bE$, $\gelm$, $\bF$, $\tau$)}{\label{scf1}
\nl		\For{$t\leftarrow 1\;\KwTo\;\tau$}{
\nl			\For{$\bp_v \in \bP$}{
\nl				\For{$\bp_i \in \bN_v$}{
					\linespread{1.5}\selectfont
\nl					$\kern-0.5pt\alpha \leftarrow cos^{-1}$
					\begin{Large}
						$\kern-6pt\left (\frac{\bE_c^2\,+\,\bE_b^2\,-\,\bE_a^2}{2\,\cdot\,\bE_c\,\cdot\,\bE_b}\right )$\tcc*[r]{Eq:~\ref{eq:alphaFromEdgeLengths}}
					\end{Large}\label{algSCFalpha}
					\linespread{1.2}\selectfont
\nl					$\kern0.00pt\beta \leftarrow (\pi - \alpha)\mathbin{/}2$\tcc*[r]{Eq:~\ref{eq:betaFromHalfAlpha}}\label{algSCFbeta}
\nl					$\kern-1.5ptA \leftarrow \Big(\gelm\,\Big)^2\kern-4pt\cdot\alpha\mathbin{/}2$\tcc*[r]{Eq:~\ref{eq:circularSectorArea}}\label{algSCFarea}
\nl					$\kern1.00pt\check{\ell} \leftarrow \big(4\cdot\gelm\cdot\sin(\alpha\mathbin{/}2)\big)\mathbin{/}3\,\alpha$\tcc*[r]{Eq:~\ref{eq:distToCoG}}\label{algSCFcog}
\nl					$\kern1.00pt\zeta \leftarrow \gelm\mathbin{/}\sin(\beta)$\tcc*[r]{Eq:~\ref{eq:zeta}}\label{algSCFzeta}
\nl					\For{$j \in {1,2}$}{\label{algSCFjloop}
\nl						$\tilde{\ell}_j \leftarrow \zeta\mathbin{/}\bE_j$\tcc*[r]{Eq:~\ref{eq:distanceIForInterpolation},~\ref{eq:distanceIp1ForInterpolation}}
\nl						$f'_j \leftarrow f_0\cdot(1 - \tilde{\ell}_j) + f_j\cdot\tilde{\ell}_j$\tcc*[r]{Eq:~\ref{eq:interpolatedFi},~\ref{eq:interpolatedFip1}}
					}
\nl					$\check{f} \leftarrow f_0\cdot(1 - \check{\ell}) + \big((f'_1 + f'_2)\cdot\check{\ell}\big)\mathbin{/}2$\tcc*[r]{Eq:~\ref{eq:weightedMeanAtCoGatSector}}\label{algSCFcheckf}
\nl					$\kern-2.0pt\tilde{f}_v \leftarrow \tilde{f}_v + A\cdot\check{f}$\tcc*[r]{Eq:~\ref{eq:meanFuncValAtPv}}\label{algSCFtildef}
\nl					$\kern-4.0pt\tilde{A}_v \leftarrow \tilde{A}_v + A$\tcc*[r]{Eq:~\ref{eq:meanFuncValAtPv}}\label{algSCFtildeA}
				}
\nl				$f'_v \leftarrow \tilde{f}_v\mathbin{/}\tilde{A}_v$\tcc*[r]{Eq:~\ref{eq:meanFuncValAtPv}}\label{algSCFfprimev}
			}
\nl		$\bF' \leftarrow \left \{f'_1,\ldots,\,f'_{|\bP|}\right \}$\;\label{algSCF2ndlastLine}
\nl 	$\bF \leftarrow \bF'$\tcc*{smooth newest values every convolution}\label{algSCFlastLine}
		}
	}
	\caption{Serial algorithm for convolving \Fors{t}\label{alg:serialConvolveFilter}}
\end{algorithm}%
\nomenclature[pa]{$\tilde{f}$}{the total volume of function values over $\bO_v$}%
\nomenclature[pb]{$\tilde{A}$}{the area of $\bO_v$}%

While the performance of Algorithm~\ref{alg:serialConvolveFilter} is much improved by the neighborhood building and pre-calculations performed in Algorithms~\ref{alg:serialBuildNeighborhoods} and~\ref{alg:serialCalculateEdgeLengths}, its strictly serial design prevents it from scaling in performance appropriately for the sizes of real-world, acquired \tdd{}, performing especially sluggishly for high numbers of convolutions on meshes with large amounts of triangulated points\todoReference{a specific experiment}; the very targets for which \Fors{t} is primarily intended.

%
%
%
%
\section{Summary}
\label{ch4sDN}
In this chapter, we presented the serial algorithms for implementing the improved version of \Fors{t}, as implemented within the GigaMesh framework. In Section~\ref{ch4sATP}, we discussed the motivation behind separating the algorithm into three parts, citing the substantial improvements to the over all efficiency of the filter. In Section~\ref{ch4sBN}, the considerations behind building a family of sets of neighborhoods are weighed, before presenting the function \textit{serialBuildNeighborhoods} in Algorithm~\ref{alg:serialBuildNeighborhoods}. Then in Section~\ref{ch4sCEL}, the significance of $\ellstar$ and the L2-norm calcuation is explained in context with modern implementations of the square root operation, before presenting function \textit{serialCalculateEdgeLengths} in Algorithm~\ref{alg:serialCalculateEdgeLengths}.   Then finally, in Section~\ref{ch4sCF}, Algorithm~\ref{alg:serialConvolveFilter} is presented with the function \textit{serialConvolveFilter} which implements the principle loop for convolving the filter over a mesh, for as many number of convolutions as required to achieve the desired smoothing effect.

Unfortunately, \Fors{t} is still yet entirely serial in design, therefore, we will in the next section, endeavor to explore this as-yet-unpublished algorithm in order to discover any manifestations of independent procedures worthy of exploiting with parallel processing, with the goal of improving the overall performance and scalability of the filter when it is implemented on a system capable of parallel computation.

