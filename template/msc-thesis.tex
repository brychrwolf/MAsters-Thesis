% This template was initially provided by Dulip Withanage.
% Modifications for the database systems research group
% were made by Conny Junghans,  Jannik Strötgen and Michael Gertz

\documentclass[
     12pt,         % font size
     a4paper,      % paper format
     BCOR10mm,     % binding correction
     DIV14,        % stripe size for margin calculation
%     liststotoc,   % table listing in toc
%     bibtotoc,     % bibliography in toc
%     idxtotoc,     % index in toc
%     parskip       % paragraph skip instad of paragraph indent
     ]{scrreprt}

%%%%%%%%%%%%%%%%%%%%%%%%%%%%%%%%%%%%%%%%%%%%%%%%%%%%%%%%%%%%

% PACKAGES:

% Use German :
\usepackage[ngerman]{babel}
% Input and font encoding
\usepackage[latin1]{inputenc}
\usepackage[T1]{fontenc}
% Index-generation
\usepackage{makeidx}
% Einbinden von URLs:
\usepackage{url}
% Special \LaTex symbols (e.g. \BibTeX):
%\usepackage{doc}
% Include Graphic-files:
\usepackage{graphicx}
% Include doc++ generated tex-files:
%\usepackage{docxx}
% Include PDF links
%\usepackage[pdftex, bookmarks=true]{hyperref}

% Fuer anderthalbzeiligen Textsatz
\usepackage{setspace}

% hyperrefs in the documents
\usepackage[bookmarks=true,colorlinks,pdfpagelabels,pdfstartview = FitH,bookmarksopen = true,bookmarksnumbered = true,linkcolor = black,plainpages = false,hypertexnames = false,citecolor = black,urlcolor=black]{hyperref} 
%\usepackage{hyperref}

%%%%%%%%%%%%%%%%%%%%%%%%%%%%%%%%%%%%%%%%%%%%%%%%%%%%%%%%%%%%

% OTHER SETTINGS:

% Pagestyle:
\pagestyle{headings}

% Choose language
\newcommand{\setlang}[1]{\selectlanguage{#1}\nonfrenchspacing}


\begin{document}

% TITLE:
\pagenumbering{roman} 
\begin{titlepage}


\vspace*{1cm}
\begin{center}
\vspace*{3cm}
\textbf{ 
\Large Universit�t Heidelberg\\
\smallskip
\Large Institut f�r Informatik\\
\smallskip
\Large Arbeitsgruppe Datenbanksysteme\\
\smallskip
}

\vspace{3cm}

\textbf{\large Master-Arbeit} % Bachelor-Arbeit 

\vspace{0.5\baselineskip}
{\huge
\textbf{Titel}
}
\end{center}

\vfill 

{\large
\begin{tabular}[l]{ll}
Name: & Name der Autorin/des Autors\\
Matrikelnummer: & Matrikelnummer  der Autorin/des Autors\\
Betreuer: & Name der Betreuerin / des Betreuers\\
Datum der Abgabe: & \today
\end{tabular}
}

\end{titlepage}

\onehalfspacing

\thispagestyle{empty}

\vspace*{100pt}
\noindent
Ich versichere, dass ich diese Master-Arbeit selbstst�ndig verfasst und nur die angegebenen
Quellen und Hilfsmittel verwendet habe und die Grunds�tze und
Empfehlungen ``Verantwortung in der Wissenschaft'' der Universit�t Heidelberg beachtet wurden. 

\vspace*{50pt}
\noindent

\underline{\phantom{mmmmmmmmmmmmmmmmmmmm}}

\medskip
\noindent 
Abgabedatum: \today
\newpage

% Add a brief summary of your topic and contributions (Zusammenfassung) in German *and* in English:
\chapter*{Zusammenfassung}

% This file contains the German version of your abstract, with about 300-500 words

Die Zusammenfassung muss auf Deutsch \textbf{und} auf Englisch geschrieben
werden. Die Zusammenfassung sollte zwischen einer halben und einer
ganzen Seite lang sein. Sie soll den Kontext der Arbeit, die
Problemstellung, die Zielsetzung und die entwickelten Methoden sowie
Erkenntnisse bzw.~Ergebnisse �bersichtlich und verst�ndlich
beschreiben.
\newpage

\chapter*{Abstract}

% This file contains an abstract of your thesis, with approximaltely 300-500 words

The abstract has to be given in German \textbf{and} English. It should
be between half a page and one page in length. It should cover in a
readable and comprehensive style the context of the thesis, the
problem setting, the objectives, and the methods developed in this
thesis as well as key insights and results.

\newpage

% MAIN PART:
% Table of contents (Inhaltsverzeichnis)
\tableofcontents
\cleardoublepage
\pagenumbering{arabic} 

% List of figures (Abbildungsverzeichnis):
%\listoffigures
% List of tables (Tabellenverzeichnis):
%\listoftables

%%%%%%%%%%%%%%%%%%%%%%%%%%%%%%%%%%%%%%%%%%%%%%%%%%%%%%%%%%%%%%%
% Here, the actual content of your thesis begins
% You can either put all the text here or use individual files to store the chapters of your thesis.
% Below are templates for both alternatives.

\chapter{Einleitung}\label{intro}

Dieses Kapitel gibt einen �berblick �ber die Arbeit. Gerade der
Abschnitt zur Motivation soll allgemein verst�ndlich geschrieben
werden. Die Einleitung sollte auch wichtige Referenzen enthalten. 

\section{Motivation}

Worum geht es? Beispiel(e)! Illustrationen sind hier meist sinnvoll
zum Verst�ndnis. Warum ist das Thema wichtig? In welchem Kontext?

\section{Ziele der Arbeit}

In diesem Abschnitt sollen neben den Herausforderungen und der
Problemstellung insbesondere die Ziele der Arbeit beschrieben werden. 

\section{Aufbau der Arbeit}

Dieser Abschnitt wird meist recht kurz gehalten und beschreibt im
Prinzip nur den Aufbau des Rests der Arbeit. Zum Beispiel: In Kapitel
\ref{chap:grundlagen} geben wir einen �berblick �ber die  Grundlagen
zu der Arbeit sowie �ber verwandte Arbeiten. In Kapitel
\ref{chap:main} stellen wir dann \ldots vor. \ldots etc.

%%%%%%%%%%%%%%%%%%%%%%%%%%%%%%%%%%%%%%%%%%%%%%%%%%%%%%%%%%%%
\newpage

\chapter{Grundlagen und verwandte Arbeiten}
\label{chap:grundlagen}

Die ersten paar Abschnitte in diesem Kapitel f�hren in die Grundlagen
zur Arbeit ein. Das k�nnen beispielsweise Grundlagen zu Netzwerken
oder zur Informationsextraktion sein. 

\section{(Beispiel) Netzwerke}
\label{sec:networks}

\section{(Beispiel) Informationsextraktion}
\label{sec:ie}

\section{Verwandte Arbeiten}
\label{sec:related}

Typischerweise im letzten Abschnitt dieses Kapitels wird dann auf
verwandte Arbeiten eingegangen. Entsprechende Arbeiten sind geeignet
zu zitieren. Beispiel: Die wurde erstmalig in den Arbeiten von Spitz
und Gertz \cite{Spitz2016a} gezeigt \ldots Details dazu werden in
dem Buch von Newman zu Netzwerken \cite{Newman2010} erl�utert \ldots.

%%%%%%%%%%%%%%%%%%%%%%%%%%%%%%%%%%%%%%%%%%%%%%%%%%%%%%%%%%%%
\newpage

% Alternative: put content in separate files
% Check the difference between including these files using \input{filename} and \include{filename} and see which one you like better
%\chapter{Einleitung}\label{intro}
%\input{introduction}
%
%\chapter{Voraussetzungen}\label{bg}
%\input{background}

%%%%%%%%%%%%%%%%%%%%%%%%%%%%%%%%%%%%%%%%%%%%%%%%%%%%%%%%%%%%
\newpage

\chapter{Mein Beitrag}
\label{chap:main}

Dieses Kapitel stellt meist den Hauptteil der Arbeit dar. Vor dem
ersten Abschnitt sollte ein kurzer �berblick (ein paar wenige S�tze
mit Verweise auf nachfolgende Abschnitte) gegeben werden. Beispiel: Im
nachfolgenden Abschnitt \ref{sec:overview} wir ein �berblick �ber die Anforderungen an
das Modell gegeben. 

\section{�berblick und Zielsetzung}
\label{sec:overview} 

Knapp zwei Seiten, in dem die Anforderungen, die Zielsetzung und die
Methoden �berblicksartig beschrieben werden. Hier sollte die
Beschreibung ``technischer'' bzw.~``formaler'' sein als in der
Einleitung, da der Leser nun mit den Grundlagen und verwandten
Arbeiten vertraut ist.

\section{Erster Teil}
\label{sec:teil1}

In diesem und den nachfolgenden Abschnitten werden die Beitr�ge der
Arbeit motiviert, formal sauber (oft mathematisch, sprich mit
Definitionen etc.) beschrieben, und bei Bedarf mithilfe von Beispielen
verdeutlicht. Die Beschreibungen in diesem Kapitel sind meist
unabh�ngig von einer konkreten Realisierung und Daten; diese werden im
nachfolgenden Kapitel detailliert.

\section{Zweiter Teil}
\label{sec:teil2}

Usw.

%%%%%%%%%%%%%%%%%%%%%%%%%%%%%%%%%%%%%%%%%%%%%%%%%%%%%%%%%%%%
\chapter{Experimentelle Evaluation}
\label{chap:eval}

Der Aufbau dieses Kapitels oder dessen Aufteilung in zwei Kapiteln ist
stark von dem Thema und der Bearbeitung des Themas
abh�ngig. Beschrieben werden hier Daten, die f�r eine Evaluation
verwendet wird (Quellen, Beispiele, Statistiken), die Zielsetzung der Evaluation
und die verwendeten Ma�e sowie die Ergebnisse (u.a.~mithilfe von
Charts, Diagrammen, Abbildungen etc.)

Dieses Kapitel kann auch mit einer Beschreibung der Realisierung eines
Systems beginnen (kein Quellcode, maximal Klassendiagramme!).

%%%%%%%%%%%%%%%%%%%%%%%%%%%%%%%%%%%%%%%%%%%%%%%%%%%%%%%%%%%%
\chapter{Zusammenfassung und Ausblick}
\label{chap:concl}

Hier werden noch einmal die wichtigsten Ergebnisse und Erkenntnisse
der Arbeit zusammengefasst (nicht einfach eine Wiederholung des
Aufbaus der vorherigen Kapitel!), welche neuen Konzepte, Methoden und
Werkzeuge Neues entwickelt wurden, welche Probleme nun (effizienter)
gel�st werden k�nnen, und es wird ein Ausblick auf weiterf�hrende
Arbeiten gegeben (z.B.~was Sie machen w�rden, wenn Sie noch 6 Monate
mehr Zeit h�tten).


% References (Literaturverzeichnis):
% a) Style (with abbreviations: use alpha):
% see
% https://de.wikibooks.org/wiki/LaTeX-W%C3%B6rterbuch:_bibliographystyle
% for the different formats and styles

\bibliographystyle{apalike}
% b) The File:
\bibliography{references}

\end{document}
