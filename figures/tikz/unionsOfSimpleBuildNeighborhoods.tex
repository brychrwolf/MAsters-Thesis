\documentclass{standalone}
\usepackage{amsmath}
\usepackage{tikz}
\usetikzlibrary{shapes,arrows.meta}
\newcommand{\fors}[1]{#1he Fast One-Ring smoothing filter}
\newcommand{\Fors}[1]{#1he Fast One-Ring smoothing filter for scalar fields on discrete manifolds}
\newcommand{\tdd}{3D-data}
\newcommand{\wmfv}[1]{weighted mean function value#1}
%
\newcommand{\bE}{\mathcal{E}}
\newcommand{\bF}{\mathcal{F}}
\newcommand{\bM}{\mathcal{M}}
\newcommand{\bN}{\mathcal{N}}
\newcommand{\bO}{\Omega}
\newcommand{\bP}{\mathcal{P}}
\newcommand{\bR}[1]{\mathbb{R}^{#1}}
\newcommand{\bT}{\mathcal{T}}
\newcommand{\bc}{\mathbf{c}}
\newcommand{\bp}{\mathbf{p}}
\newcommand{\bs}{\mathbf{s}}
\newcommand{\bt}{\mathbf{t}}
\newcommand{\bv}{\mathbf{v}}
%
\newcommand{\elm}{\ell_\text{min}}
\newcommand{\gelm}{\overline{\elm}}
\newcommand{\ellstar}{\ell_\ast}
%
\newcommand{\fM}{\mathfrak{M}}
%
\newcommand{\mbeq}{\overset{!}{=}}
%
\newcommand{\sipo}{i\kern-.7pt\scalebox{0.66}{+}\kern-1.2pt1}
\newcommand{\sipt}{i\kern-.7pt\scalebox{0.66}{+}\kern-1.2pt2}
\newcommand{\sjpo}{j\kern-.7pt\scalebox{0.66}{+}\kern-1.2pt1}
\newcommand{\sjpt}{j\kern-.7pt\scalebox{0.66}{+}\kern-1.2pt2}
\newcommand{\sps}{\kern-2pt+\kern-3pt}
\newcommand{\sxpx}[2]{#1\kern-.7pt\scalebox{0.66}{+}\kern-1.2pt#2}
\newcommand{\sv}[1]{v,\kern.75pt #1}
%
\newcommand{\todoRemove}[1]{\todo[color=red!40]{Remove: #1}}
\newcommand{\todoAsk}[1]{\todo[color=yellow!40]{Ask: #1}}
\newcommand{\todoCitation}[1]{\todo[color=teal!40]{Cite: #1}}
\newcommand{\todoReference}[1]{\todo[color=lime!40]{Ref: #1}}
\newcommand{\todoResearch}[1]{\todo[color=magenta!40]{Research: #1}}
\newcommand{\todoBackground}[1]{\todo[color=violet!40]{Bg: #1}}
\newcommand{\todoReword}[1]{\todo[color=cyan!40]{Reword: #1}}
\newcommand{\todoStyle}[1]{\todo[color=pink!40]{Style: #1}}
%xcolor base colors:
%	black
%	blue
%	brown
%%%	cyan
%%%	lime
%%%	magenta
%	olive
%	orange
%%%	pink
%	purple
%%%	red
%%%	teal
%%%	violet
%	white
%%%	yellow

\definecolor{MyTeal}{rgb}	{0, 	.5, 	.5}		%teal = 0,127,127
\definecolor{MyLtTeal}{rgb}	{.8125, .9375, 	.9375}	%lt.teal = 207,239,239
\definecolor{MySand}{rgb}	{1, 	.625, 	0}		%sand = 255,159,0
\definecolor{MyLtSand}{rgb}	{1, 	.9766, 	.875}	%lt.sand = 255,249,223
\definecolor{MyCoral}{rgb}	{1, 	.375, 	.375}	%coral = 255,96,96
\definecolor{MyLtCoral}{rgb}{1, 	.875, 	.875}	%lt.coral = 255,223,223


\begin{document}
\tikzset{%
	>={Latex[width=2mm,length=2mm]},
	baseNode/.style = {rectangle, rounded corners,
		draw=black, fill=white, thick,
		minimum width=1cm, minimum height=1cm,
		text centered, font=\sffamily, inner sep=.2cm},
	baseLine/.style = {thick},%double},
	tealStyle/.style = {draw=MyTeal, fill=MyLtTeal},
	coralStyle/.style = {draw=MyCoral, fill=MyLtCoral},
	sandStyle/.style = {draw=MySand, fill=MyLtSand},
	%
	faceL/.style = {baseNode, sandStyle},
	lineL/.style = {baseLine, sandStyle},
	faceR/.style = {baseNode, coralStyle},
	lineR/.style = {baseLine, coralStyle},
	lineC/.style = {baseLine, tealStyle},
	point/.style = {baseNode},
	nbhd/.style = {baseNode, minimum width=2cm}
}
\begin{tikzpicture}[node distance=0cm]
	\coordinate (center1) at (0cm,0cm);
	\node (t1) [faceL, anchor=east, xshift=-.5cm] {$\bt_1 = \{\bp_1,\,\bp_2,\,\bp_3\}$};
	\node (t2) [faceR, anchor=west, xshift= .5cm] {$\bt_2 = \{\bp_3,\,\bp_2,\,\bp_4\}$};
	\coordinate (center2) at (0cm,-1.5cm);
	\node (p1) [point, left of=center2, xshift=-3cm] {$\bp_1$};
	\node (p2) [point, left of=center2, xshift=-1cm] {$\bp_2$};
	\node (p3) [point, right of=center2, xshift=1cm] {$\bp_3$};
	\node (p4) [point, right of=center2, xshift=3cm] {$\bp_4$};
	\coordinate (center3) at (0cm,-4.5cm);
	\node (n1) [nbhd, left of=center3, xshift=-5cm] {$\bN_1$};
	\node (n2) [nbhd, left of=center3, xshift=-1.75cm] {$\bN_2$};
	\node (n3) [nbhd, right of=center3, xshift=1.75cm] {$\bN_3$};
	\node (n4) [nbhd, right of=center3, xshift=5cm] {$\bN_4$};

	\draw[-, lineL] (t1) -- (p1);
	\draw[-, lineL] (t1) -- (p2);
	\draw[-, lineL] (t1) -- (p3);

	\draw[->, lineL] (p1) -- (n2);
	\draw[->, lineL] (p1) -- (n3);
	\draw[->, lineL] (p2) -- (n1);
	\draw[->, lineC] (p2) -- (n3.180);
	\draw[->, lineL] (p3) -- (n1);
	\draw[->, lineC] (p3) -- (n2);

	\draw[-, lineR] (t2) -- (p2);
	\draw[-, lineR] (t2) -- (p3);
	\draw[-, lineR] (t2) -- (p4);

	\draw[->, lineC] (p3) -- (n2.0);
	\draw[->, lineR] (p3) -- (n4);
	\draw[->, lineR] (p2) -- (n4);
	\draw[->, lineC] (p2) -- (n3);
	\draw[->, lineR] (p4) -- (n3);
	\draw[->, lineR] (p4) -- (n2);
\end{tikzpicture}
\end{document}
